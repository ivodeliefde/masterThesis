%!TEX root = thesis.tex

%since specific data and tools have to be used, it’s good to present these concretely, so that the mentors know that you have a grasp of all aspects of the project;


\section{Tools and Data}
\label{chap:TD}

\subsection{Data}
Topographic data of neighbourhoods, city districts, municipalities and provinces\\

Air quality sensor data from the \ac{rivm} (\url{http://inspire.rivm.nl/sos/}) and from the \ac{ircel} (\url{http://sos.irceline.be/}). \\

\iffalse
The \ac{lustre} aims to "help users to easier and better express and use thesauri and controlled vocabularies for metadata work within Spatial Data Infrastructures" \cite[p. 137]{LD:LusTRE}. This ontology does not focus on sensor data, but does describe many environmental phenomenon that are being observed. This fits with the idea of \cite{SSW:Cox4} to use ontologies that are already available online for defining metadata elements of sensor data. Since the \ac{sos} from the \ac{RIVM} and the \ac{sos} from \ac{irceline} provides information on   
\fi

\subsection{Database}
A Postgres database will be used with the Postgis extension.

\subsection{Server}
Prototyping will be done using a localhost at first, but in the end it could be hosted on the university server.


\subsection{Prototype}
\begin{itemize}
	\item The Python programming language will be used for scripting a prototype. 
	\item Psycopg2 will be used to connect a Python script to a Postgres database.
	\item Python's \href{http://docs.python-requests.org/en/latest/user/quickstart/}{Request} library will be used for making \ac{http} POST and GET requests. 
	\item For working with \ac{xml} Python's xml package will be used.
	\item To create \ac{rdf} documents the Python library \href{https://rdflib.readthedocs.org/en/stable/}{RDFLib} will be used. 
	\item The scripts will be part of a \ac{wps} using \href{http://pywps.wald.intevation.org/}{PyWPS} 
\end{itemize}









  

