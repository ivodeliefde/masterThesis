%!TEX root = thesis.tex
\chapter{Tools and Data}
\label{chap:TD}

since specific data and tools have to be used, it’s good to present these concretely, so that the mentors know that you have a grasp of all aspects of the project;

\section{Data}
Topographic data of neighbourhoods, city districts, municipalities and provinces\\
Air quality sensor data from the \ac{rivm} (http://inspire.rivm.nl/sos/) and from the \ac{ircel} (http://sos.irceline.be/).

\section{Database}
A Postgres database will be used with the Postgis extension.

\section{Server}
Prototyping will be done using a localhost at first, but in the end it could be hosted on the university server.

\section{Prototype}
\begin{itemize}
	\item The Python programming language will be used for scripting a prototype. 
	\item Psycopg2 will be used to connect a Python script to a Postgres database.
	\item Python's \href{http://docs.python-requests.org/en/latest/user/quickstart/}{Request} library will be used for making \ac{http} POST and GET requests. 
	\item For working with \ac{xml} Python's xml package will be used.
	\item To create \ac{rdf} documents the Python library RDFLib will be used. 
\end{itemize}









  

