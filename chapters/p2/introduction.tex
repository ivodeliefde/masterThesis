%!TEX root = ../thesis.tex

\chapter{Introduction}
\label{chap:introduction}

This document should include:

\begin{itemize} 
\item motivation / problem field /relevance

\item position in the academic and professional debate

\item problem statement, objectives, research questions

\item approach, theoretical framework, methodology

\item references

\item preliminary project set up and results

\end{itemize}


an introduction in which the relevance of the project and its place in the context of geomatics is described, along with a clearly-defined problem statement\\

\cite{LD:Egenhofer} argues "there would exist a much higher potential for exploiting the Web if tools were available that better match human reasoning" \cite[p. 1]{LD:Egenhofer}. \cite{SSW:Sheth} proposed adding semantics to the sensor web which has resulted in the so-called \ac{ssw}. The \ac{swe} standards by the \ac{ogc} have been combined with ontologies in the \ac{ssw}  to add meaning to sensor data \citep{SSW:Pschorr, SSW:Henson}. The \ac{w3c} has contributed to this development by creating the \ac{ssn} ontology \citep{SSW:SSN_incubatorGroup}.

Smart cities are defined as .... \citep{SC:Townsend}. Research on 'smart cities' has identified a number of potential uses for semantic sensor data. Often this is presented in the context of the \ac{iot} \citep{IOT:Zanelli, SSW:Perera}. 

\cite{IOT:Fell} describes the \ac{iot} as "the result of technological progress in many parallel and often overlapping fields, including those of embedded systems, ubiquitous and pervasive computing, mobile telephony, telemetry and machine-to-machine communication, wireless sensor networks, mobile computing, and computer networking" \cite[p. 11]{IOT:Fell}. \cite{SSW:Perera} argues that sensor data fusion plays an important role in the \ac{iot}. He defines sensor data fusion as "a data processing technique that associates, combines, aggregates, and integrates data from different sources" \cite[p. 2]{SSW:Perera}. 

To keep up with the developments of the \ac{iot} and the smart city \ac{ogc} has identified the role of their standards in the 'smart cities spatial information framework' \cite{SC:OGC}.  

\section{problem statement}
The vision of the \ac{iot} and the smart city is very appealing. However, there are still a number of missing links considering the \ac{ssw}. First of all, research on \ac{ssw} focusses mainly on sensor data from a single source. It is therefore difficult to discover sensor data on the internet that can be queried using the \ac{swe} standards. Second of all, data from different sources cannot easily be shared, integrated, combined and aggregated for data fusion \citep{SSW:Wang, SSW:Ji, SSW:Corcho}. 

To overcome the issues of sensor data discovery and integration \cite{SSW:Janowicz} suggest a linked data model and a \ac{rest}ful proxy for \ac{ogc}’s \ac{sos}. They mention future work should focus on the 'Sensor Plug\&Play infrastructure' in which "sensors can by automatically registered" \cite[p. 21]{SSW:Janowicz}. Also, it is argued that these developments could lead to a micro-SDI and will enable a ubiquitous Geospatial Web. 

