%!TEX root = ../thesis.tex

\chapter{Introduction}
\label{chap:introduction}

From 2020 onwards all member states of the \ac{eu} should provide sensor data to the \ac{inspire} in order to comply with annex II and III of the \ac{inspire} directive \citep{SDI:INSPIRE5}. For this a number of \ac{swe} standards are required to be used \citep{SDI:INSPIRE2}. The sensor web is a relatively new development and there are still many question on how to structure it. This thesis aims to develop a method to publish and link sensor metadata on the semantic web for discovering, integrating and aggregating sensor data.

% Introduction to the subject
\section{Background}
In 2008 the \ac{ogc} introduced a new set of standards called \ac{swe}. These standards make it possible to connect sensors to the internet and retrieve data in a uniform way. This allows users or applications to retrieve sensor data through standard protocols, regardless of the type of observations or the sensor's manufacturer \citep{SW:Botts}. Among other standards \ac{swe} includes the \ac{om} which is a model for encoding sensor data, the \ac{sensorml} which is a model for describing sensor metadata and the \ac{sos} which is a service for retrieving sensor data \citep{SW:OGC}. \ac{om} has also been adopted by the \ac{iso} under \ac{iso} 19156:2011 \citep{SW:ISO}. 

Recently \ac{ogc} has defined the role which their standards could play in smart city developments \citep{SC:OGC}. Smart cities can be defined as "enhanced city systems which use data and technology to achieve integrated management and interoperability" \citep[p. 18]{SC:Moir}. Research on smart cities has shown a great potential for using sensor data in urban areas. Often this is presented in the context of the \ac{iot} \citep{IOT:Zanelli, SSW:Wang2}. The \ac{iot} can be described as "the pervasive presence around us of a variety of \textit{things} or \textit{objects} ... [which] are able to interact with each other and cooperate with their neighbors to reach common goals" \cite[p. 2787]{IOT:Atzori}. 

Parallel to the development of the sensor web other research has focused on the semantic web, as proposed by \cite{LD:Berners-lee}. This is a response to the traditional way of using the web, where information is only available for humans to read. The semantic web is an extension of the internet which contains meaningful data that machines can understand as well. Rather than publishing documents on the internet the semantic web contains linked data using the \ac{rdf}, also known as the web of data \citep{LD:Bizer}. The semantic web was primarily intended to add metadata on the internet \cite{LD:W3C}. However, today it is being used for linking any kind of data from one source to another in a meaningful way \citep{LD:Cambridge}. 

\cite{SSW:Sheth} proposes to use semantic web technologies in the sensor web. This so-called \ac{ssw} builds on standards by \ac{ogc} and the \ac{w3c} "to provide enhanced descriptions and meaning to sensor data" \cite[p.78]{SSW:Sheth}. \ac{w3c} responded to this development by developing a standard ontology for sensor data on the semantic web \citep{SSW:SSN_incubatorGroup}. 
 
% Problem statement
\section{Problem statement}
Finding sensor data that can be retrieved using a \ac{sos} is not easy. The implementation of the sensor web is still in an early stage. There are only a limit number of \ac{sos} implementations available on the web and they contain a limited amount of data. For example, the \ac{sos} by te \ac{rivm} has only recently been created and contains only data on air quality. A number of companies and organisations still use a custom \ac{api} to connect sensors to the internet. On top of that there is no extensive register or catalogue service for discovering sensor data via a \ac{sos}, something which is possible for services such as \ac{wms}, \ac{wfs} or \ac{wcs} \citep{SDI:OGC2}.

It has been argued that it is difficult to integrate sensor data from different sources to perform data fusion \citep{SSW:Corcho, SSW:Ji, SSW:Wang}. Data fusion is "a data processing technique that associates, combines, aggregates, and integrates data from different sources" \cite[p. 2]{SSW:Wang2}. Even if a number of sensor data sources are available that use the \ac{swe} standards it is challenging, since the data can be of a different scale, both in time and space. 

A question that comes to mmind is to what extent the semantic web could be a better solution for publishing sensor data than using a \ac{sos}. The geoweb has a very structured approach in which sensor data can be retrieved using well defined services. This is different from for example web pages where content is completely unstructured. The response of a \ac{sos} also contains semantics about sensor data. There can be x-links inside the \ac{xml} with \ac{uri}s that point to semantic definitions of objects. Still, the semantic web could be beneficial for the geoweb as it is machine understandable which could be useful for automatic integration and aggregation. It also contains links to other relevant data which could make discovering sensor data more easy.   

% Research question 
\section{Research question}
This thesis aims to develop a method that uses the semantic web to improve sensor data discovery as well as the integration and aggregation of sensor data from heterogeneous sources. The following question will be answered in this research:   
\textit{How can the semantic web improve the discovery, integration and aggregation of distributed sensor data?} 












































\iffalse

This document should include:

\begin{itemize} 
\item motivation / problem field /relevance

\item position in the academic and professional debate

\item problem statement, objectives, research questions

\item approach, theoretical framework, methodology

\item references

\item preliminary project set up and results

\end{itemize}


\section{Use Case}
Providing temperature data of all EU member states is part of the \ac{inspire} program. This data is useful for mapping the effects of extreme temperatures on citizen's health and the environment. \cite{UC:vanderHoeven} and \cite{UC:vanderHoeven2} are examples of studies which aim to identify urban heat islands in cities. These studies require temperature data combined with statistics, based on addresses. This research will be used as an example use case for the added value of the semantic sensor web. 
 

\fi