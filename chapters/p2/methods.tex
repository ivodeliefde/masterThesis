%!TEX root = thesis.tex
\chapter{Methods}
\label{chap:methods}

overview of the methodology to be used;
\iffalse
\section{Standards}
\subsection{Sensor Web Enablement}
\ac{sos} and \ac{ssn}

\subsection{Semantic Web}

\begin{itemize}
	\item Store \ac{osm} data: create \ac{rdf} on-the-fly to prevent double storage. 
	\item Retrieve sensor metadata: create \ac{rdf} from \ac{om} which is returned by \ac{sos} getCapabilities/describeSensor requests
	\item Use \ac{ssn} as ontology for semantic sensor metadata 
	\item Use \ac{owl} as language for storing semantic \ac{rdf} triples
\end{itemize}


Query metadata: SPARQL, geo-SPARQL or stSPARQL? and which query engine?

\section{Ontologies}
The \ac{om} ontology is retrieved from sensors\\
Use \ac{ssn} ontology ontology to store metadata from sensors.  

\section{Middleware}
Creating own middleware to link sensors semantically and retrieve aggregated data\\

\ac{rest}ful service

\begin{figure}
	\centering
	\includegraphics[width=2\linewidth, angle=90]{figs/flowchart.png}
	\caption{Flow chart showing the differen components of the implementation}
	\label{fig:Flowchart}
\end{figure}
\fi




