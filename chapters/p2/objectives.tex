%!TEX root = ../thesis.tex

\section{Research objectives}
\label{chap:objectives}

%the research objectives and/or research questions are clearly defined, along with the scope (ie what you will not be doing);\\

\subsection{Research question} 
The main question this thesis will try to answer is: 

\textit{To what extent can the semantic web improve the discovery, integration and aggregation of distributed sensor data?}\\

To answer the main question a number of sub-questions need to be answered:
\begin{itemize}
	\item To what extent can sensor metadata be automatically retrieved from a \ac{sos} and published on the semantic web?
	\item How can metadata on the semantic web be linked to relevant features-of-interest using existing vocabularies?
	\item To what extent can incoming links be automatically generated from DBPedia?
	\item How can aggregation methods be represented on the semantic web to formalise expert knowledge and prevent meaningless aggregation?
	\item To what extent can already existing standards for retrieving geographic data be used for a service that supplies integrated and aggregated sensor data?
\end{itemize}

\subsection{Objectives}

This thesis explores a method to store metadata of sensors on the semantic web, and to link it to real world features-of-interest and to appropriate methods for aggregation. This should improve the discovery of sensor data through links to other related data on the internet.  

To improve the integration of sensor data a middleware architecture will be designed that can return sensor data for features-of-interest from different sources. The sensor data can be returned raw or aggregated. Only appropriate methods of aggregation are offered for each kind of observations, based on a formalisation of expert knowledge on the semantic web. The service can be accessed as a \ac{wps}. A prototype implementation will be created as a proof of concept.

\subsection{Scope}
The focus of this thesis is to explore the role of the semantic web in discovering, integrating and aggregating sensor data from heterogeneous sources. Therefore, accepted \ac{ogc} and \ac{iso} standards will be, such as \ac{sos}, \ac{sensorml} and \ac{om}. The om-lite and sam-lite ontologies will be used in combination with geoSPARQL. An in-depth analysis of different (upcoming) standards lies outside the scope of this research. 


\subsection{Research methodology}

\hl{Paragraph that describes that the research area is the Netherlands and Belgium.}\\
\hl{The data should be retrieved, but also the SOS services}\\
\hl{The prototype application returns more than the raw data, it processes it first (integrating and aggregating) }\\
\hl{For online processes is an OGC standard: WPS. The prototype will be a WPS.}\\

The idea by \cite{SW:Jones} of delivering data to users through a service with which they are already familiar is very appealing, because it enables data to be immediately used in any existing \ac{gis}. This is also suggested by \cite{SSW:Atkinson} to ease the transition to the linked data world. However, current research has mainly been concerned with static geographic data, not with sensor data. Therefore, this thesis aims to provide a service for integrating and aggregating sensor data as a \ac{wps}.



