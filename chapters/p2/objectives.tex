%!TEX root = ../thesis.tex

\chapter{Research Objectives}
\label{chap:objectives}

the research objectives and/or research questions are clearly defined, along with the scope (ie what you will not be doing);\\

\section{Research Question} 
The main question this thesis will try to answer is: 

How can the semantic sensor web improve the discovery, integration and 
aggregation of distributed sensor data?\\

To answer the main question a number of sub-questions need to be answered:
\begin{itemize}
	\item How can sensor metadata be retrieved from a \ac{sos} and published on the semantic web with links to features-of-interest in an automated process?
	\item How can aggregation methods be represented on the semantic web to formalise expert knowledge and prevent meaningless aggregation?
	\item What are the advantages and disadvantages of integrating sensor data from different sources?
	\item To what extent can already existing standards for retrieving geographic data be used for a service that supplies integrated and aggregated sensor data?
\end{itemize}

\section{Objectives}

This thesis will explore a method that stores metadata of sensors on the semantic web, and links it to real world features of interest and to appropriate methods for aggregation. This should improve the discovery of sensor data through links to other related data on the internet.  

To improve the integration of sensor data a middleware architecture will be developed that can return sensor data for features-of-interest. The returned sensor data will be aggregated. Only appropriate methods of aggregation are offered for each kind of observations, based on formalised expert knowledge on the semantic web.

\section{Scope}


\iffalse

\section{Research Question} 
How can the semantic sensor web improve the discovery, integration and aggregation of distributed sensor data?


\section{Objective} 
Develop a method to semantically link sensor metadata to real world objects for spatial, semantic and temporal data aggregation.\\

I would like to bridge the efforts by \cite{SSW:Henson}, \cite{SSW:Pschorr},  of adding semantics to \ac{sos}, and the efforts by \cite{LD:Auer} of adding semantics to \ac{osm} data using the \ac{ssn} ontology proposed by \cite{SSW:SSN_incubatorGroup} in order to improve the discovery, integration and aggregation of sensor data from different sources.

The thesis research should result in a prototype implementation which consists of two parts: 
\begin{enumerate}
	\item Firstly, an application that takes locations (HTTP addresses) of SOS servers as input and automatically links them to the \ac{osm} data. It results in a(n extended) mapping of the sensor web that will be used by the aggregation queries in the second part of the implementation. However, these mappings between sensors and objects are also freely accessible on the (semantic) web which improves the discovery of sensors for other applications.
	\item Secondly, an application that allows users to query aggregated sensor data from different sources. This takes an \ac{osm} feature and a time interval as input, optionally with other spatial/temporal parameters (like a value for a buffer operation or a time interval to aggregate on). It returns a set of aggregated sensor data. 
\end{enumerate}


\section{Scope} 
focus on \ac{ogc}'s \ac{swe} standards / \ac{sos} and \ac{w3c}'s \ac{ssn} ontology. Not going into evaluation of different standards. Not specifically focussing on smart devices, but on \ac{swe} enabled sensors.
\fi