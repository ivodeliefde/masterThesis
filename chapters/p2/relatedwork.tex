%!TEX root = thesis.tex
\chapter{Related work}
\label{chap:rw}

a related work section in which the relevant literature is presented and linked to the project;
\section{Semantic sensor data middleware}
\cite{SSW:Henson} and \cite{SSW:Pschorr} suggest adding semantic annotations to a \ac{sos} which they call semantically enabled \ac{sos} or Sem-SOS. In Sem-SOS the raw sensor data goes through a process of semantic annotating before it can be requested with a \ac{sos} service. The retrieved data is still an \ac{xml} document, but with embedded semantic terminology as defined in an ontology model. 

\cite{SSW:Janowicz} has specified a method that uses a \ac{rest}ful proxy as a fa\c{c}ade for \ac{sos}. When a specific \ac{uri} is requested the so-called \ac{sel} translates this to a \ac{sos} request, fetches the data and translates the results back to \ac{rdf}. In this method the sensor data is converted to \ac{rdf} on-the-fly.  

\section{Sensor data ontologies}
\cite{SSW:SSN_incubatorGroup} have developed an ontology based on the stimulus-sensor-observation pattern.

\cite{SSW:Cox3} has been working on an improved semantic ontology based on \ac{om}.

\section{Sensor data aggregation}
\cite{SSW:Stasch} propose to aggregate sensor data based on the geometry of sampling features. \cite{SSW:Stasch4} argue that in order for automatic aggregation to work there needs to be semantics on which kind of aggregation methods are appropriate for a specific kind of sensor data. This requires a formalisation of expert knowledge which they call semantic reference systems.
 

\iffalse
\section{Sensor Web}

\ac{ogc} \ac{swe} standards such as \ac{sos} and \ac{sensorml} \citep{SW:OGC}, \citep{SW:Botts} \\

Different data formats: \ac{xml} (\ac{swe}), EXI (\ac{w3c}) and \ac{json} SensorThings \ac{api} \citep{IOT:Zanelli} \\

\ac{ogc} has been working on standards for geodata discovery that allow users to find datasets based on geographic and temporal properties \citep{SW:OGC2}. It is available as an extension for existing formats like RSS and Atom. However, it does not mention support for sensor data discovery. \\

Sensor metadata visualisation has been studied by \cite{SW:Yoo} in order to improve the discovery of sensor data. How to integrate or combine different datasets is not addressed in this approach. 


\section{Linked Data}

Linked Data \citep{LD:Berners-lee}

Web of Data \cite{LD:Bizer} 

Research by \cite{LD:Auer} under the name of 'linked geodata' shows how to convert \ac{osm} data into \ac{rdf} triplets and publish it in the semantic web. The linked \ac{osm} data used to be accessible online at \url{http://linkedgeodata.org/OnlineAccess}. However, this is no longer the case. They do offer a manual and open source software for publishing \ac{osm} data as linked data at \url{https://github.com/GeoKnow/LinkedGeoData}. Anyone interested in using linked \ac{osm} data can refer to this for their own implementation.     

Publishing geodata as \ac{rdf} and mapping on-the-fly \citep{LD:Missier}


\section{Semantic Sensor Web}

\ac{ssw} \citep{SSW:Sheth}, \citep{SSW:deMel}, \citep{SSW:Bakillah} \\ 

\cite{SSW:Huang} add semantics to sensor data, but do not use the \ac{swe} standards by \ac{ogc}. \\ 

Extending RDF with the ability to represent spatial and temporal data \citep{SSW:Koubarakis} \\ 

\cite{SSW:Janowicz2} argue that the current \ac{swe} standards are not sufficient for dealing with sensor data on the semantic web. A transparent \ac{sel} for \ac{ogc} services is proposed that replaces the \ac{om} and \ac{sensorml} standards. This research formed the basis for the development of the \ac{w3c} \ac{ssn} ontology \citep{SSW:SSN_incubatorGroup}. \\ 

\cite{SSW:Henson} and \cite{SSW:Pschorr} present the SemSOS architecture which adds semantics to sensor data and publishes the semantically enriched sensor data via \ac{sos} and the \ac{om} encoding. However, in SemSOS the semantics are not being used to integrate or combine data from different sources, but rather to enrich data from a single source. \\ 

Applications using \ac{ogc} \ac{sos} and \ac{w3c} \ac{ssn} \citep{SSW:Kessler}, \citep{SSW:Barnaghi} \\

Three layer model: the sensor data source layer, the data integration layer and the application layer \cite{SSW:Wang}.

\section{Internet of Things}

More and more devices connected to the internet. Also a growing amount of research on using sensors of smart devices. \citep{IOT:Waher}, \citep{IOT:Zarko}, \citep{SSW:Calbimonte} \\

Research on connecting smart devices to \ac{ssw} \citep{SSW:Vera}  \\

OpenIoT platform \citep{IOT:Calbimonte}


\section{Smart Cities} 

The role of sensors in smart cities \citep{IOT:Zanelli}

The role of \ac{ogc} standards in smart cities \citep{SC:OGC}
\fi
