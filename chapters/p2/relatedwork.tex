%!TEX root = thesis.tex
\chapter{Related work}
\label{chap:rw}

a related work section in which the relevant literature is presented and linked to the project;


\section{Sensor Web}
Research related to making sensor data available via the internet. \\

\ac{ogc} \ac{swe} standards such as \ac{sos} and \ac{sensorml} \citep{SW:OGC}, \citep{SW:Botts} \\

Different data formats: \ac{xml} (\ac{swe}), EXI (\ac{w3c}) and \ac{json} SensorThings \ac{api} \citep{IOT:Zanelli} \\

\ac{ogc} has been working on standards for geodata discovery that allow users to find datasets based on geographic and temporal properties \citep{SW:OGC2}. It is available as an extension for existing formats like RSS and Atom. However, it does not mention support for sensor data discovery. \\

Sensor metadata visualisation has been studied by \cite{SW:Yoo} in order to improve the discovery of sensor data. How to integrate or combine different datasets is not addressed in this approach. 


\section{Linked Data}
Research related to creating machine understandable (semantic) knowledge on the internet \\

Linked Data \citep{LD:Berners-lee}

Web of Data \cite{LD:Bizer} 

Research by \cite{LD:Auer} under the name of 'linked geodata' shows how to convert \ac{osm} data into \ac{rdf} triplets and publish it in the semantic web. The linked \ac{osm} data used to be accessible online at \url{http://linkedgeodata.org/OnlineAccess}. However, this is no longer the case. They do offer a manual and open source software for publishing \ac{osm} data as linked data at \url{https://github.com/GeoKnow/LinkedGeoData}. Anyone interested in using linked \ac{osm} data can refer to this for their own implementation.     

Publishing geodata as \ac{rdf} and mapping on-the-fly \citep{LD:Missier}


\section{Semantic Sensor Web}
Research related to adding semantic meaning to sensors data allowing more complex queries \\ 

\ac{ssw} \citep{SSW:Sheth}, \citep{SSW:deMel}, \citep{SSW:Bakillah} \\ 

\cite{SSW:Huang} add semantics to sensor data, but do not use the \ac{swe} standards by \ac{ogc}. \\ 

Extending RDF with the ability to represent spatial and temporal data \citep{SSW:Koubarakis} \\ 

\ac{w3c} \ac{ssn} ontology \citep{SSW:SSN_incubatorGroup} \\ 

\cite{SSW:Henson} and \cite{SSW:Pschorr} present the SemSOS architecture which adds semantics to sensor data and publishes the semantically enriched sensor data via \ac{sos} and the \ac{om} standard (which is also part of the \ac{ogc} \ac{swe} suite). However, in SemSOS the semantics are not being used to integrate or combine data from different sources, but rather to enrich data from a single source. \\ 

Research on connecting smart devices to \ac{ssw} \citep{SSW:Vera}  

Three layer model: the sensor data source layer, the data integration layer and the application layer \cite{SSW:Wang}.

\section{Internet of Things}
Research related to connecting smart devices to the internet (which are often mobile devices with sensors) to allow devices to communicate with each other \\

More and more devices connected to the internet. Also a growing amount of research on using sensors of smart devices. \citep{IOT:Waher}, \citep{IOT:Zarko}, \citep{SSW:Calbimonte} \\


OpenIoT platform \citep{IOT:Calbimonte}


\section{Smart Cities}
Research related to the application of sensors networks for improving the quality of life in a city \\ 

The role of sensors in smart cities \citep{IOT:Zanelli}

The role of \ac{ogc} standards in smart cities \citep{SC:OGC}
