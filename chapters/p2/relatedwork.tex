%!TEX root = thesis.tex
% a related work section in which the relevant literature is presented and linked to the project;
\chapter{Related work}
\label{chap:rw}
A number of research topics are relevant for this thesis: how to connect existing standards for publishing sensor data to the semantic web, developing ontologies that are suitable for many different kinds of sensor data and how to aggregate sensor data based on features-of-interest and time. This chapter discusses the recent relevant literature on these topics.     

\section{Semantic sensor data middleware}
\cite{SSW:Henson} and \cite{SSW:Pschorr} suggest adding semantic annotations to a \ac{sos} which they call \ac{semsos}. In \ac{semsos} the raw sensor data goes through a process of semantic annotating before it can be requested with a \ac{sos} service. The retrieved data is still an \ac{xml} document, but with embedded semantic terminology as defined in an ontology model. The data retrieved from \ac{semsos} is therefore semantically enriched.  

\cite{SSW:Janowicz} has specified a method that uses a \ac{rest}ful proxy as a fa\c{c}ade for \ac{sos}. When a specific \ac{uri} is requested the so-called \ac{sel} translates this to a \ac{sos} request, fetches the data and translates the results back to \ac{rdf}. In this method the sensor data is converted to \ac{rdf} on-the-fly. This allows the data to be interpreted by both humans and machines.  

\cite{SSW:Atkinson} have identified that "distributed heterogeneous data sources are a necessary reality in the case of widespread
phenomena with multiple stakeholder perspectives" \cite[p.129]{SSW:Atkinson}. Therefore, they propose that methods should be developed to move away from the traditional dataset centric approaches and towards using linked data for cataloguing. This has the potential to bring together data and knowledge from different areas of research about the same (or similar) features-of-interest. It is also argued that using both linked data services and data-specific services could ease the transition into the linked data world.  

\section{Sensor data ontologies}
 \ac{w3c} has developed an ontology based on the stimulus-sensor-observation pattern \citep{SSW:SSN_incubatorGroup}.

\cite{SW:Hu} has reviewed a number of metadata models (including \ac{sensorml} and \ac{ssno}). They argue that all of the current metadata models are not sufficient for sensor data discovery. Therefore, they propose a metadata model that "reuses and extends the existing sensor observation-related metadata standards" \cite[p. 10546]{SW:Hu}.

\cite{SSW:Cox3} has been working on an improved semantic ontology based on \ac{om}.

\section{Sensor data aggregation}
\cite{SSW:Stasch} propose to aggregate sensor data based on the geometry of sampling features. \cite{SSW:Stasch3} proposes a \ac{wps} that takes sensor data right from a \ac{sos} service in order to aggregate it. The approach by \cite{SSW:Stasch} takes sensor data as input that is already published on the semantic web.

\cite{SSW:Stasch4} argue that in order for automatic aggregation to work there needs to be semantics on which kind of aggregation methods are appropriate for a specific kind of sensor data. This requires a formalisation of expert knowledge which they call semantic reference systems.

\section{Conclusion}
\ac{semsos} \citep{SSW:Henson, SSW:Pschorr} as well as \ac{sel} \citep{SSW:Janowicz} focus on combining the sensor web with the semantic web, but do not address the integration and aggregation of sensor data. Similarly, \cite{SSW:Atkinson} proposes to expose sensor data to the semantic web in order to find other kinds of related data about the same feature-of-interest. Data that is collected from another area of research for example. Also \cite{SSW:Atkinson} does not mention the integration of sensor data from heterogeneous sources. \cite{SSW:Stasch} and \cite{SSW:Stasch3} suggest interesting methods for aggregating sensor data based on features-of-interest. However, also these studies use sensor data from a only single source into account. Moreover, \cite{SSW:Corcho} and \cite{SSW:Ji} argue that methods for integration and fusion of sensor data on the semantic web is still an area for future research. Data fusion is "a data processing technique that associates, combines, aggregates, and integrates data from different sources" \cite[p. 2]{SSW:Wang2}. This thesis therefore focuses on the discovery, integration and aggregation of sensor data, building on some of the principles proposed by related research discussed in this chapter.  

The idea by \cite{SW:Jones} of delivering data to users through a service with which they are already familiar is very appealing, because it would enable sensor data to be immediately used in any existing \ac{gis}. This is also suggested by \cite{SSW:Atkinson} to ease the transition to the linked data world. However, current research has mainly been concerned with static geographic data, not with (aggregated) sensor data. Therefore, this thesis aims to provide the service for integrating and aggregating sensor data as a \ac{wps}.

\iffalse
\section{Sensor Web}

\ac{ogc} \ac{swe} standards such as \ac{sos} and \ac{sensorml} \citep{SW:OGC}, \citep{SW:Botts} \\

Different data formats: \ac{xml} (\ac{swe}), EXI (\ac{w3c}) and \ac{json} SensorThings \ac{api} \citep{IOT:Zanelli} \\

\ac{ogc} has been working on standards for geodata discovery that allow users to find datasets based on geographic and temporal properties \citep{SW:OGC2}. It is available as an extension for existing formats like RSS and Atom. However, it does not mention support for sensor data discovery. \\

Sensor metadata visualisation has been studied by \cite{SW:Yoo} in order to improve the discovery of sensor data. How to integrate or combine different datasets is not addressed in this approach. 


\section{Linked Data}

Linked Data \citep{LD:Berners-lee}

Web of Data \cite{LD:Bizer} 

Research by \cite{LD:Auer} under the name of 'linked geodata' shows how to convert \ac{osm} data into \ac{rdf} triplets and publish it on the semantic web. The linked \ac{osm} data used to be accessible online at \url{http://linkedgeodata.org/OnlineAccess}. However, this is no longer the case. They do offer a manual and open source software for publishing \ac{osm} data as linked data at \url{https://github.com/GeoKnow/LinkedGeoData}. Anyone interested in using linked \ac{osm} data can refer to this for their own implementation.     

Publishing geodata as \ac{rdf} and mapping on-the-fly \citep{LD:Missier}


\section{Semantic Sensor Web}

\ac{ssw} \citep{SSW:Sheth}, \citep{SSW:deMel}, \citep{SSW:Bakillah} \\ 

\cite{SSW:Huang} add semantics to sensor data, but do not use the \ac{swe} standards by \ac{ogc}. \\ 

Extending RDF with the ability to represent spatial and temporal data \citep{SSW:Koubarakis} \\ 

\cite{SSW:Janowicz2} argue that the current \ac{swe} standards are not sufficient for dealing with sensor data on the semantic web. A transparent \ac{sel} for \ac{ogc} services is proposed that replaces the \ac{om} and \ac{sensorml} standards. This research formed the basis for the development of the \ac{w3c} \ac{ssn} ontology \citep{SSW:SSN_incubatorGroup}. \\ 

\cite{SSW:Henson} and \cite{SSW:Pschorr} present the SemSOS architecture which adds semantics to sensor data and publishes the semantically enriched sensor data via \ac{sos} and the \ac{om} encoding. However, in SemSOS the semantics are not being used to integrate or combine data from different sources, but rather to enrich data from a single source. \\ 

Applications using \ac{ogc} \ac{sos} and \ac{w3c} \ac{ssn} \citep{SSW:Kessler}, \citep{SSW:Barnaghi} \\

Three layer model: the sensor data source layer, the data integration layer and the application layer \cite{SSW:Wang}.

\section{Internet of Things}

More and more devices connected to the internet. Also a growing amount of research on using sensors of smart devices. \citep{IOT:Waher}, \citep{IOT:Zarko}, \citep{SSW:Calbimonte} \\

Research on connecting smart devices to \ac{ssw} \citep{SSW:Vera}  \\

OpenIoT platform \citep{IOT:Calbimonte}


\section{Smart Cities} 

The role of sensors in smart cities \citep{IOT:Zanelli}

The role of \ac{ogc} standards in smart cities \citep{SC:OGC}
\fi
