
\chapter*{Abstract}


Developments such as smart cities, the \acf*{iot} and the \ac*{inspire} are producing a growing amount of observation data. The \ac*{ogc} has developed \ac*{swe} standards for modelling and publishing this data online. However, their use is currently limited to geo information specialists, who have knowledge about which data services are available and how to access them. With the use of the semantic web, online processes can automatically find and understand observation metadata. This opens up the \acs*{swe} services to a large user audience. Therefore, this thesis has designed a conceptual system architecture that uses the semantic web to improve sensor data discovery as well as the integration and aggregation of sensor data from multiple sources.

The presented conceptual system architecture contains two web processes. The first process automatically creates an online semantic knowledge base of sensor metadata, by harvesting \aclp*{sos}. The metadata is based on the \acf*{om} data model and includes which sensors are deployed, what they observe, how they observe it and at which \acs*{swe} service their data can be requested. The second process can be used to automatically translate logical queries of users into observation data requests. It also performs further processing before returning the observation data to the user. These two processes have been tested in a proof of concept implementation.

A \ac*{wps} has been created as a proof of concept, making the two processes available online. The proof of concept is able to harvest sensor metadata, convert it to linked data, and publish it on the semantic web with links to and from other metadata. The created semantic knowledge base improves the discovery of observation data, and enables it to be machine understandable. The \acs*{wps} is therefore able to retrieve and process observation data from multiple \aclp*{sos}, using input parameters, such as: \textit{What are the average particulate matter levels per month in neighbourhoods of Delft over the last five years?}  
 
Creating a completely automated process for harvesting metadata, which works with every \acl*{sos} is not yet feasible, as non-meaningful identifiers are allowed to be used according to the \acs*{swe} standards. Linked data ontologies should also be extended to include all available metadata. Currently there are no ontologies able to define the concept of a geo web service. The final conclusion is that the \acs*{sparql} endpoints used in this thesis cannot cope with complex vector geometries. This is due to their inability to handle verbose queries. Based on these conclusions, five recommendations are presented.