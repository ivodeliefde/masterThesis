
\chapter*{Abstract}


Developments such as smart cities, the \acf*{iot} and the \ac*{inspire} are causing a growing amount of observation data to be produced. The \ac*{ogc} has developed \ac*{swe} standards for modelling and publishing this data online. However, their use is currently limited to geo information specialists, who have knowledge about which data services are available and how to access them. With the use of the semantic web, online processes can automatically find and understand observation metadata. This opens up the \acs*{swe} services to a large user audience. Therefore, this thesis has designed a conceptual system architecture that uses the semantic web to improve sensor data discovery as well as the integration and aggregation of sensor data from multiple sources.

A conceptual system architecture is presented containing two web processes. The first process automatically creates an online semantic knowledge base of sensor metadata, by harvesting \aclp*{sos}. Metadata based on the \acf*{om} data model are retrieved, which includes the location of deployed sensors, what they observe, how they observe it and at which \acs*{swe} service their data can be requested. The second process automatically translates logical queries of users into observation data requests. It also performs further processing before returning the observation data to the user. Both of these processes have been tested in a proof of concept implementation.

A \ac*{wps} has been created as a proof of concept, making the two processes available online. The proof of concept is able to harvest sensor metadata, convert it to linked data, and publish it on the semantic web with links to and from other metadata. It results in a semantic knowledge base of sensor metadata, which improves the discovery in a machine understandable way. It allows multiple \aclp*{sos} to be used in combination with each other, due to the harmonisation involved in creating linked data. The \acs*{wps} is therefore able to retrieve and process observation data from multiple \aclp*{sos}, using queries such as: \textit{What are the average particulate matter levels per month in neighbourhoods of Delft over the last five years?}  
 
In conclusion, the presented conceptual system architecture contains two processes which improve the discovery, retrieval and aggregation of sensor data from multiple \aclp*{sos} using the semantic web. However, creating a completely automated process for harvesting metadata, which works with every \acl*{sos} is not yet feasible, as non-meaningful identifiers are allowed to be used according to the \acs*{swe} standards. Linked data ontologies should also be extended to include all available metadata. Currently there are no ontologies able to define the concept of a geo web service. Another conclusion is that the \acs*{sparql} endpoints used in this thesis cannot cope with complex vector geometries. This is due to their inability to handle verbose queries. Nevertheless, the proof of concept shows that the semantic web can enable current \ac{swe} services to be used in new ways, making observation data available to a larger user audience.  