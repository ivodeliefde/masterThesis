% !TeX root = ../../thesis.tex

% ~2 pages 
\chapter{Conclusions}
\label{chap:conclusion}

This thesis aimed to design a conceptual system architecture that uses the semantic web to improve sensor data discovery as well as the integration and aggregation of sensor data from multiple sources. Each of the four subquestions posed in the introduction will be answered in this chapter (Paragraph \ref{subquestions}), before answering the main research question \ref{mainQuestion}.

\section{Subquestions}
\label{subquestions}
\begin{enumerate}
\item \textit{To what extent can sensor metadata be automatically retrieved from any SOS?}%\mbox{}

Data inside a \acl{sos} can be automatically retrieved from a \ac{sos}, using the methods described in Paragraph \ref{chap:retrieveSOS}. However, there are two things in the \ac{swe} standards that should be improved to make the proposed design work better. 

\begin{sloppypar}
	First of all, the capabilities document is currently only required to contain a list of all features of interest as a parameter for the \texttt{GetObservation} request. Optionally, the features of interest can also be mentioned as metadata per offering. In the case of air quality these features represent the sensor locations. However, it is merely required to list the \acp{uri} of the features of interest. To retrieve their geometries either \texttt{GetFeatureOfInterest} or \texttt{GetObservation} requests have to be made. This is especially cumbersome since the \texttt{GetFeatureOfInterest} response does not link features of interest to procedures or observed properties. This relation is therefore only visible by either combining the observed properties and procedures per offering to the \texttt{GetFeatureOfInterest} response or by requesting observations from each sensor location. Therefore, I propose that the capabilities document should not only list the \acp{uri} of the features of interest, but also describe the geometry and observed properties of each of them. 
\end{sloppypar}

Secondly, in the current specification of the \ac{xml} schemas for \ac{sos} and \ac{om}, identifiers for features of interest, observed properties and procedures can be \textit{any \ac{uri}}. This means, that an \ac{url} with semantics can be provided, but a non-semantic \ac{uri} would also be valid. I propose to make the standard more strict and require a semantic \ac{url} that can be resolved to an \ac{rdf} document. Without these semantics it is hard to use a \ac{sos} for both humans and automatic processes, especially when two or more are used in combination with each other. 

Besides the two changes which should be made to the \ac{swe} standards, there is another issue with automatically retrieving sensor metadata from a \ac{sos}: the order of coordinates for a feature of interest. For example, both \aclp{sos} used in this thesis had a different order for the latitude and the longitude coordinates. This is a standardisation issue already identified by many authors, which should be decided upon by the geomatics and geoscience community. For the method described in this thesis it is irrelevant which order is being used, as long as its clearly described in individual cases or prescribed using an international standard.

\item \textit{To what extent can sensor metadata from a SOS be automatically converted to linked data and published on the semantic web?}%\mbox{}

In a \ac{sos} there are \ac{xml} schemas which contain general semantics about the metadata. They identify what the different \acp{uri} represent (e.g. observed properties, procedures, features of interest). Using a workflow adapted from \cite{LD:Missier}, this metadata can be automatically converted to linked data. The different elements in the \ac{xml} schemas have been mapped to the corresponding classes in linked data ontologies to allow a \ac{wps} to create \ac{rdf} documents, and publish them at a \ac{sparql} endpoint. The om-lite, sam-lite and PROV ontologies have been used to describe the linked metadata.

However, a number of classes should be added to these ontologies to make it suit this process better. First of all, a class is required that distinguishes the \textit{process} of creating an observation from the physical \textit{device} that uses this process. This \texttt{sensor} class could be modelled as a device that uses a procedure at a certain sampling point. Adding this class takes away some of the ambiguity between defined processes and actually deployed sensors. Therefore, better semantic reasoning can be implemented to return metadata about deployed sensors, of which data can be retrieved.  

Another class that should be represented in an ontology is the \acl{sos}. The current prototype design (Chapter \ref{chap:impl}) used a single endpoint for storing all metadata from \aclp{sos}. Therefore, it was known that the data source is a \ac{sos}, but it has not been properly defined in a linked data ontology. If programs crawling the semantic web can identify a data source such as a \ac{sos} and understand its supported queries (e.g. \texttt{GetCapabilities}, \texttt{DescribeSensor}, \texttt{GetObservation}), they can retrieve data without requiring prior knowledge on \ac{swe} standards. This way, sensor data using other platforms such as the SensorThings \ac{api} could be discovered and retrieved in the same way, and used in combination with each other. Therefore the extensions of current ontologies is further discussed as future research in Chapter \ref{chap:futureResearch}.  
           

\item \textit{What is an effective balance between the semantic web and the geo web in the chain of discovering, retrieving and processing sensor data?}%\mbox{}

Triple stores do not perform as good as non-\ac{rdf} relational databases when large amounts of data are being queried \citep{LD:Bizer2}. On the other hand, linked data is very well suited for discovering data as it is literally \textit{linked} to related data. Therefore, this thesis aimed to design a method for using the semantic web in combination with sensor web applications, where the semantic web contains metadata and the geoweb observation data. However, there is a grey area of functionality that could be implemented using either one of these two parts of the web. 

On the one hand, the semantic web could have a bounding box per \ac{sos} containing all features of interest it offers in combination with a list of all observed properties. In this case spatial and temporal filters would have to be applied at the \ac{sos} side when retrieving observation data. On the other hand, the semantic web could also contain detailed information about individual sensors. This way the \aclp{sos} are only used to retrieve observation data of already selected sensors. Both approaches have been considered.

In the proof of concept a more detailed semantic knowledge base is created for a number of reasons. First of all, it was found that not all \aclp{sos} offer the same filter capabilities. For the first option to be viable every \ac{sos} should have a minimum amount of filter capabilities implemented by default. 
Secondly, if a user is interested in sensors located in a spatial feature, the bounding box of all sensors might be misleading. The bounding box is a rough generalisation of the \ac{foi} and could include sensors that are not overlapping the vector geometry of the \ac{foi}. The result of this is that many unnecessary requests will be send to a \ac{sos}. This lowers the overall performance, as the client has to wait for more requests to be executed, and the \ac{sos} server as to handle requests from more clients. 

Thirdly, semantic information about specific sensors can be linked by other related linked data. This can be done by the organisation maintaining the sensor or by other organisations. For example, links to the manufacturer, the quality of observations achieved by a certain model of sensors, or the conditions under which the sensor is placed could be useful for anyone interested in the observation data. Information like this could all be added to a semantic knowledge base and does not have to be provided by a \ac{sos}.

However, the down side of the a detailed semantic knowledge base is that more data has to be transferred over the internet. First a \ac{sparql} has to be made with a spatial, which includes verbose \ac{wkt} geometries. Then an \ac{rdf} document containing detailed information about each sensor is return, based on which \ac{sos} requests are performed. If only the \aclp{sos} are returned that have sensors inside a certain bounding box, both the \ac{sparql} request and response are smaller. The detailed spatial query is then performed by the \aclp{sos}. 

Still, discovering sensors is only a matter of seconds using the a detailed semantic knowledge base. Automatically retrieving sensor data can take up to a couple minutes depending on the amount of sensors for which data is requested and the temporal range. However, it should be noted that performance optimization is out of the scope of this thesis. It is likely that this can still be improved significantly (see Chapter \ref{chap:futureResearch}).              

\item \textit{To what extent can already existing standards for retrieving data be (re)used for a service that supplies integrated and aggregated sensor data?}%\mbox{}

All data models and services in this thesis have been used because they are based on open standards. Designs for two processes have been explored: an automated process for creating linked data from metadata in a \ac{sos} and a process for discovering, retrieving and processing sensor data. These processes were created using \ac{ogc} \aclp{wps}, which is a standard \ac{api} for spatial data processes on the web. The \ac{wps} is well suited for these two applications.

The \ac{om} standard could be reused on the semantic web using the om-lite and sam-lite ontologies. These are lightweight linked data ontologies based on \ac{om}. Performing spatial queries on this linked data is possible using \ac{ogc}'s GeoSPARQL as well as using Strabon's stRDF (Paragraph \ref{par:SpatialFilters}).

The \ac{om} observation schema is being reused for the spatially aggregated sensor data which is retrieved from the discovered \aclp{sos}. However, the schema only allows (aggregated) sensor data from the same procedure. The result of this is, that only observations from the same \ac{sos} fit in the \ac{om} observation schema, as different procedures were implemented by the different organisations maintaining a \ac{sos}. Therefore, I propose to allow the procedure element to contain an array of multiple procedures from which the data originates. 
\end{enumerate}

\section{Main question}
\label{mainQuestion}
\begin{quote}
	\textit{To what extent can the semantic web improve the discovery, integration and aggregation of distributed sensor data?}%\mbox{}
\end{quote}

In this thesis a conceptual system architecture has been designed to create an online knowledge base with linked metadata extracted from \aclp{sos}. This helps discovering, integrating and aggregation sensor data, while for efficient data retrieval the \ac{sos} is still used. The results show that such a knowledge base makes it easier to discover sensors and their corresponding sources. Compared to a \ac{csw}, the discovery is improved, because it contains semantic descriptions of sensors, and there are inward links by another large semantic knowledge base: DBPedia. For example, a client can find sensors which observe NO$_{2}$ levels of their \ac{foi} via the description of NO$_{2}$ on DBPedia. This description contains links to related metadata, which are required for creating \texttt{GetObservation} requests.

Only open standards have been used in the conceptual system architecture. They have been preferred over commercial ones, as they are based consensus in the geosciences community and are free to be (re)used by anyone. Any \ac{sos} can be added to the semantic knowledge base, if there are a minimum of semantics provided. The data sources used in the proof of concept did not meet this criteria, because the \ac{sos} specifications are to some extent, open for interpretation. A manual step had to be added to the workflow to cope with this. Therefore, the \ac{swe} standards should be extended to allow the presented method to be completely automated. 

Primarily experts in the field of geo-information have knowledge on \ac{swe} standards and data services. The semantic knowledge base enables online processes to translate a logical query for sensor data into technical queries. This makes sensor data accessible to a larger audience, who might not be familiar with \acp{uri}, encodings, data models, service models or specific data formats. With sensor metadata as linked data, a user only needs to enter parameters, such as: the type of sensor data, a spatial feature and a temporal range. For example: \textit{What are the average particulate matter levels per month in neighbourhoods of Delft over the last five years?}. A \ac{wps} can automatically translate this into \texttt{GetObservation} requests using the semantic knowledge base. The presented proof of concept implementation also sends the requests to the discovered \aclp{sos}, integrates their responses and performs further processing such as data aggregation. The user receives a single data set from the \ac{wps}, according to the logical data query.   


\section{Recommendations}

Based on the results of this thesis a number of recommendations are formulated. 

\begin{itemize}
	\item Create more strict specifications for \acp{uri} in \aclp{sos}
	
	Observed properties and procedures need to be semantically defined by the organisation maintaining a \ac{sos}, which is not mandatory using the current \ac{swe} standards specifications. Requiring semantic \acp{uri} to be used which can be resolved to \ac{rdf} documents would force \aclp{sos} to be machine understandable.
	
	\item Enable the \ac{om} Observation to include aggregated sensor data with the same observed property, created by different sensing procedures. 
	
	The output of the proof of concept implementation is not in accordance with the \ac{om} \ac{xml} schema, because it outputs aggregated observation data. This data can be retrieved from different sources, each using different procedures. Therefore, multiple procedures should be an allowed value for the procedure element in the \ac{om} Observation schema.   
	
	\item Extend linked data ontologies with classes for geo web services, supported requests and \ac{sos} content.
	
	The om-lite and sam-lite ontologies cover a large part of the sensor metadata. However, these linked data ontologies should be extended to define the \ac{sos} service, it's supported requests and additional metadata, such as observation offerings. 
	
	\item Improve the performance of endpoints, to better cope with vector geometries in \ac{sparql} queries
	
	Since GeoSPARQL and stSPARQL use the verbose \ac{wkt} or \ac{gml} encodings the queries can be rejected for exceeding the maximum amount of characters. For this reason it is found that the performance of the endpoint should be improved, to better cope with vector geometries in \ac{sparql} queries. 
	
\end{itemize}




