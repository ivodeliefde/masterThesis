% !TeX root = ../../thesis.tex

% ~2 pages 
\chapter{Conclusions}
\label{chap:conclusion}

This thesis aimed to design a method that uses the semantic web to improve sensor data discovery as well as the integration and aggregation of sensor data from multiple sources. Each of the subquestions posed in the introduction will be answered followed by the main research question.

\paragraph{To what extent can sensor metadata be automatically retrieved from a SOS?}\mbox{}

Data inside a \acl{sos} can be automatically converted to linked data. However, there are two things in the \ac{swe} standards that should be improved to make the proposed design work better. 

First of all, the capabilities document is currently only required to contain a list of all features of interest as a parameter for the \texttt{GetObservation} request. Optionally the features of interest can also be mentioned as metadata per offering. In the case of air quality these represent the sensor locations. However, it is merely required to list the \acp{uri} of the features of interest. To retrieve their geometries either \texttt{GetFeatureOfInterest} or \texttt{GetObservation} requests have to be made. This is especially cumbersome since the \texttt{GetFeatureOfInterest} response does not link features of interest to procedures or observed properties. This relation is only visible by either combining the observed properties and procedures per offering to the \texttt{GetFeatureOfInterest} response or by requesting observations from this sensor location. Therefore, I propose that the capabilities document should not only list the \acp{uri} of the features of interest, but also describe the geometry and observed properties of each feature. 

Secondly, in the current implementation of the schema's for \ac{sos} (\url{http://schemas.opengis.net/sos/2.0}) and \ac{om} (\url{http://schemas.opengis.net/om/2.0/}) identifiers for features of interest, observed properties and procedures can be `any \ac{uri}'. This means that an \ac{url} with semantics can be provided, but a non-semantic \ac{uri} would also be valid. I propose to make the standard more strict and require a semantic \ac{url} that can be resolved to an \ac{rdf} document. Without these semantics it is hard to use a \ac{sos} for both humans and automatic processes. Paragraph \ref{par:publishLD} describes how to use a \ac{purl} server for creating persistent semantic \acp{url} that resolve to an \ac{rdf} document. 

\paragraph{To what extent can sensor metadata from a SOS be automatically published on the semantic web?}\mbox{}

In this thesis the om-lite and sam-lite ontologies have been used to make linked data from the \ac{sos} metadata. However, a number of classes should be added to these ontologies to make it suit this process better. First of all, a class that distinguishes the process of creating an observation from the physical device that uses this process. This `sensor' class could be modelled as a device that uses a procedure at a certain sampling point. Adding this class takes away some of the ambiguity between theoretical processes and actually deployed sensors. Therefore, it will be easier to perform (spatial) \ac{sparql} queries that return deployed sensors of which data can be retrieved.  

Another class that should be represented in an ontology is the \acl{sos}. The current prototype design (Chapter \ref{chap:impl}) used a single endpoint for storing all metadata from \aclp{sos}. Therefore, it was known that the source of data is a \ac{sos}, but it has not properly defined in a linked data ontology. If programs crawling the semantic web can identify a data source such as a \ac{sos} and understand its allowed queries (\texttt{GetCapabilities}, \texttt{GetObservation}, etcetera), they can retrieve data from it without requiring prior knowledge. Therefore the extensions of current ontologies is further discussed as future research in Chapter \ref{chap:futureResearch}.  

%Semantically defining the \ac{sos} and its allowed queries opens the door to automatically combining, analysing or processing data from different \ac{ogc} geo web services. If all \ac{ogc} geo web services and their queries are part of an ontology a \ac{gis} would be able to use these services without having built-in knowledge on how they work. Future versions of a services or the addition of a service could take place by extending the ontology instead of updating each installation of a \ac{gis}.             

\paragraph{What is an efficient balance between the semantic web and the geo web in the chain of discovering, retrieving and processing sensor data?}\mbox{}



\paragraph{To what extent can already existing standards for retrieving data be (re)used for a service that supplies integrated and aggregated sensor data?}\mbox{}

All data models and services in this thesis have been used because they are based on open standards. Designs for two processes have been explored: an automated process for creating linked data from metadata in a \ac{sos} and a process for discovering, retrieving and processing sensor data. These processes were created using \ac{ogc} \aclp{wps}, which is a standard \ac{api} for spatial data processes on the web. \ac{om} could be reused on the semantic web using the om-lite ontology. Performing spatial queries on this linked data is possible using \ac{ogc}'s GeoSPARQL.

When sensor data is retrieved and aggregated from the discovered \aclp{sos} the \ac{om} observation schema is being reused. However, this only works with sensor data from a single procedure. The result of this is that only observations from the same \ac{sos} could be used, as different procedures were implemented by the organisations maintaining the \ac{sos}. Therefore, I propose 


\paragraph{To what extent can the semantic web improve the discovery, integration and aggregation of distributed sensor data?}\mbox{}

An online knowledge base with linked metadata extracted from \aclp{sos} makes sensor data easy to discover. It allows for automatic data retrieval and processing. It can be generated from any SOS, if there are a minimum of semantics provided. 
