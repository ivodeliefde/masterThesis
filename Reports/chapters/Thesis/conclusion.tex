% !TeX root = ../../thesis.tex

% ~2 pages 

\chapter{Conclusions}
\label{chap:conclusion}

This thesis aimed to design a conceptual system architecture that uses the semantic web to improve sensor data discovery as well as the integration and aggregation of sensor data from multiple sources. Each of the four subquestions posed in the introduction will be answered, before answering the main research question (Section \ref{subquestions}). The second section discusses the contributions of this thesis (Section \ref{contributions}), followed by a brief discussion (Section \ref{discussion}). At the end of the chapter, five recommendations are presented, based on the conclusions (Section \ref{recommend}).

\section{Answering the research question}
\label{subquestions}

First the four subquestions will be answered, followed by the main research question: To what extent can the semantic web improve the discovery, integration and aggregation of distributed sensor data?

\begin{enumerate}
\item \textit{To what extent can sensor metadata be automatically retrieved from any SOS?}%\mbox{}

Data inside a \ac{sos} can be automatically retrieved, using the methods described in Subsection \ref{chap:retrieveSOS}. However, there are two things in the \ac{swe} standards that should be improved to make the proposed design work better. 

\begin{sloppypar}
	First of all, the capabilities document is currently only required to contain a list of all \acp{foi} as a parameter for the \texttt{GetObservation} request. In the case of air quality these features represent the sensor locations. To retrieve their geometries either \texttt{GetFeatureOfInterest} or \texttt{GetObservation} requests have to be made. This is especially cumbersome since the \texttt{GetFeatureOfInterest} response is not required to link \acp{foi} to procedures or observed properties. This fundamental relation between a sensor location, procedure and observed property is therefore only visible by either combining \texttt{GetCapabilities}, \texttt{GetFeatureOfInterest} and \texttt{DescribeSensor} responses, or by requesting observations from sensor locations. Therefore, I propose that the capabilities document should not only list the \acp{uri} of the \acp{foi}, but also describe the geometry and observed properties of each of them. 
\end{sloppypar}

Secondly, in the current specification of the \acsu{xml} schemas for \ac{sos} and \ac{om}, identifiers for \acp{foi}, observed properties and procedures can be \textit{any \ac{uri}}. This means, that a \ac{url} with semantics can be provided, but a non-semantic \ac{uri} would also be valid. I propose to make the standard more strict and require a semantic \ac{url} that can be resolved to an \ac{rdf} document. Without these semantics it is hard to use a \ac{sos} for both humans and automatic processes, especially when two or more services are used in combination with each other. 

Besides the two changes which should be made to the \ac{swe} standards, there is another issue with automatically retrieving sensor metadata from a \ac{sos}: the order of coordinates for a \ac{foi}. For example, both \aclp{sos} used in this thesis had a different order for the latitude and the longitude coordinates. This is a standardisation issue already identified by many authors, which should be decided upon by the geomatics and geoscience community. For the method described in this thesis it is irrelevant which order is being used, as long as its clearly described in individual cases or prescribed using an international standard.

Concluding, sensor metadata can be automatically retrieved from any \ac{sos}. In the proof of concept implementation this process has also been successfully created. However, it could be further improved by implementing the above mentioned changes to the \ac{swe} standards.

\item \textit{To what extent can sensor metadata from a SOS be automatically converted to linked data and published on the semantic web?}%\mbox{}

In a \ac{sos} there are \ac{xml} schemas which contain general semantics about the metadata. They identify what the different \acp{uri} represent (e.g. observed properties, procedures, features of interest). Using a workflow adapted from \cite{LD:Missier}, this metadata can be automatically converted to linked data. The different elements in the \ac{xml} schemas have been mapped to the corresponding classes in linked data ontologies. This allows a \ac{wps} to create \ac{rdf} documents, and publish them in a \acs{sparql} endpoint. The om-lite, sam-lite and PROV ontologies have been used to describe the linked metadata.

However, a number of classes should be added to these ontologies to make it suit this process better. First of all, a class is required that distinguishes the \textit{process} of creating an observation from the physical \textit{device} that uses this process. This \texttt{sensor} class could be modelled as a device that uses a procedure at a certain sampling point. Adding this class takes away some of the ambiguity between defined processes and actually deployed sensors. Therefore, better semantic reasoning can be implemented to return metadata about deployed sensors, of which data can be retrieved.  

Another class that should be represented in an ontology is the \acl{sos}. The proof of concept (Chapter \ref{chap:impl}) used a single endpoint for storing all metadata from \aclp{sos}. Therefore, it was known that the data source is a \ac{sos}, but it has not been properly defined in a linked data ontology. If programs crawling the semantic web can identify a data source such as a \ac{sos} and understand its supported queries (e.g. \texttt{GetCapabilities}, \texttt{DescribeSensor}, \texttt{GetObservation}), they can retrieve data without requiring prior knowledge on \ac{swe} standards. Similarly, sensor data using other platforms such as the SensorThings \acs{api} could be discovered and retrieved, and used in combination with each other.           

\item \textit{What is an effective balance between the semantic web and the geoweb in the chain of discovering, retrieving and processing sensor data?}%\mbox{}

Triple stores do not perform as good as non-\ac{rdf} relational databases when large amounts of data are being queried \citep{LD:Bizer2}. On the other hand, linked data is very well suited for discovering data as it is literally \textit{linked} to related data. Therefore, this thesis aimed to design a method for using the semantic web in combination with sensor web applications, where the semantic web contains metadata and the geoweb observation data. However, there is a grey area of functionality that could be implemented using either one of these two parts of the web. 

On the one hand, the semantic web could have a bounding box per \ac{sos} containing all features of interest it offers in combination with a list of all observed properties. In this case spatial and temporal filters would have to be applied at the \ac{sos} side when retrieving observation data. On the other hand, the semantic web could also contain detailed information about individual sensors. This way the \aclp{sos} are only used to retrieve observation data of already selected sensors. Both approaches have been considered.

In the proof of concept a more detailed semantic knowledge base is created for a number of reasons. First of all, it was found that not all \aclp{sos} offer the same filter capabilities. For the first option to be viable every \ac{sos} should have a minimum amount of filter capabilities implemented by default. 
Secondly, if a user is interested in sensors overlapping spatial features, using the bounding box might not be very efficient. The bounding box is a rough generalisation of the \acp{foi} and could include sensors that are not overlapping the intended features. The result of this is that many unnecessary requests will be send to a \ac{sos}. This lowers the overall performance, as the client has to wait for more requests to be executed, and the \ac{sos} server has to handle requests from more clients. 

Thirdly, semantic information about specific sensors can be linked by other related linked data. This can be done by the organisation maintaining the sensor or by other organisations. For example, links to the sensor manufacturer, the quality of observations achieved by a certain model of sensors, or the conditions under which the sensor is placed, could be useful for anyone interested in the observation data. Information like this could all be added to a semantic knowledge base and does not have to be provided by (the organisation maintaining) a \ac{sos}. When there are no detailed descriptions of individual sensors this kind of information could not be included in the semantic knowledge base. 

However, the down side of the detailed semantic knowledge base is that more data has to be transferred over the internet. First a \acs{sparql} query has to be made with a spatial filter, which includes verbose \ac{wkt} geometries. Then an \ac{rdf} document containing detailed information about each sensor is returned, based on which \ac{sos} requests are performed. If only the addresses of \aclp{sos} are returned, which have sensors inside a certain bounding box, both the \acsu{sparql} requests and responses are smaller. The detailed spatial query is then performed by the \aclp{sos} and does not require as much data to be transferred over the internet, which makes it a more efficient procedure. 

Still, discovering sensors is only a matter of seconds in the proof of concept. On the other hand, automatically retrieving observation data from \aclp{sos} can take up to a couple minutes, depending on the amount of sensors for which data is requested and the temporal range. However, it should be noted that performance optimisation is beyond the scope of this thesis. It is likely that this can still be improved significantly in the future (see Chapter \ref{chap:futureResearch}).              

%\clearpage

\item \textit{To what extent can already existing standards for retrieving data be (re)used for a service that supplies integrated and aggregated sensor data?}%\mbox{}

All data models and services in this thesis have been used because they are based on open standards. Designs for two processes have been explored: an automated process for creating linked data from metadata in a \ac{sos} and a process for discovering, retrieving and processing sensor data. These processes were created using \ac{wps}, which is a standard \ac{api} for data processes on the web by the \ac{ogc}. The \ac{wps} is well suited for these two applications.

The \ac{om} data model could be reused on the semantic web using the om-lite and sam-lite ontologies. These are lightweight linked data ontologies based on \ac{om}. Performing spatial queries on this linked data is possible using \ac{ogc}'s GeoSPARQL as well as using Strabon's stRDF (Section \ref{par:SpatialFilters}). These are both open standards for spatial query functions in \acs{sparql}.

The \ac{om} Observation schema is being reused for the spatially aggregated sensor data, which is created from the data of discovered \aclp{sos}. However, the schema only allows sensor data with a single procedure description. As different procedures were implemented by the different organisations maintaining a \ac{sos}, only aggregated observations from the same \ac{sos} fit in the \ac{om} observation schema. The proof of concept integrates and aggregates data from different sources. Therefore, its output is not in accordance with the schema. I propose to allow the procedure element to contain a description of multiple procedures from which the data originates, to facilitate the integration of different sensor data sources. 
\end{enumerate}


\subsection*{\textit{To what extent can the semantic web improve the discovery, integration and aggregation of distributed sensor data?}}%\mbox{}


In this thesis a conceptual system architecture has been designed to create an online knowledge base with linked metadata extracted from \aclp{sos}. This helps discovering, integrating and aggregation sensor data, while for efficient data retrieval the \ac{sos} is still used. The results show that such a knowledge base makes it easier to discover sensors and their corresponding sources. Different \aclp{sos} can be integrated, because their content is being semantically harmonised in the process of creating linked data, with the use of a \ac{wps}. Having linked data of \ac{sos} metadata also allows web processes to perform tasks automatically, such as data aggregation.

Only open standards have been used in the conceptual system architecture. They have been preferred over commercial ones, as they are based on consensus in the geosciences community and they are free to be (re)used by anyone. Any \ac{sos} can be harmonised and added to the semantic knowledge base, if there are a minimum of semantics provided. The data sources used in the proof of concept did not meet this criteria, because the \ac{sos} specifications are to some extent open for interpretation. A manual step had to be added to the workflow to cope with this. Therefore, the \ac{swe} standards should be extended to allow the presented method to be completely automated. 

Primarily experts in the field of geo-information have knowledge on \ac{swe} standards and data services. The semantic knowledge base enables online processes to translate a logical queries for sensor data into technical queries. This makes sensor data accessible to a larger audience, who might not be familiar with \acp{uri}, encodings, data models, service models or specific data formats. With sensor metadata as linked data, a user only needs to enter parameters, such as: the type of sensor data, a spatial feature and a temporal range. For example: \textit{What are the average particulate matter levels per month in neighbourhoods of Delft over the last five years?} A \ac{wps} can automatically translate this into \texttt{GetObservation} requests using the semantic knowledge base. The presented proof of concept has implemented this and automatically sends the requests to the discovered \aclp{sos}, integrates their responses and performs further processing such as data aggregation. The user receives a single data set from the \ac{wps}, according to the logical data query.   

\section{Contributions}
\label{contributions}

The contribution of this thesis is a conceptual system architecture, including:
\begin{enumerate}
	\item A process for harvesting and harmonising \ac{sos} metadata, which are stored as linked data in a semantic knowledge base.
	\item A process for discovering, retrieving and processing observation data by translating a logical query to \ac{sos} requests, using a semantic knowledge base with linked sensor metadata. 
\end{enumerate}

The semantic knowledge base improves the following sensor web functionalities:\\
\begin{addmargin}[0.5cm]{0cm}
	\textbf{Discovery}: Metadata is linked to and by related data, making it easier to find online.\\
	\textbf{Integration}: Through a process of metadata harmonisation, data of different \aclp{sos} can be used in combination with each other.  \\
	\textbf{Aggregation}: A semantic knowledge base with linked sensor metadata allows web processes to automatically perform tasks, such as spatial and temporal aggregation. 
\end{addmargin}


\section{Discussion}
\label{discussion}

\ac{sos} metadata is stored twice in two different locations using the methods of this thesis. On the one hand, having duplicated data may not be desirable, because the duplicates can become out of sync with the corresponding original data at the source. Also, more storage space is required for the same amount of data. On the other hand, extra functionality is achieved when duplicating metadata to a semantic knowledge base (e.g. discovery, integration and aggregation). The \ac{csw} as well creates duplicate data of metadata from \ac{ogc} geo web services. The \acs{sir} extension for the \ac{csw} has a fixed time interval that can be set, for performing regular updates. The current method presented in this thesis has not yet implemented an update mechanism. It could be integrated into common \ac{sos} software packages, to (only) update when metadata changes at the source. 

The conceptual system architecture of Chapter \ref{chap:design} aims to automate as many steps in the chain of discovering, retrieving and processing sensor data as possible. This includes a harvesting process which should be executed to automatically create linked data from \ac{sos} metadata. It is more convenient for users, because it requires less knowledge, it appeals to a broader audience and it makes less mistakes in repetitive tasks. However, with an automated process, the quality of the metadata in a \ac{sos} influences the quality of the metadata at the \ac{sparql} endpoint. Since the method is designed to work with any \ac{sos} as input there lies a responsibility on the side of the organisations maintaining these services to provide sufficient, understandable and correct metadata.   

Spatial features have topological relations with other spatial features. These relations can be made explicit on the semantic web using ontologies such as GeoSPARQL. However, in this thesis they have not been made explicit and are calculated on-the-fly with spatial queries, using stSPARQL filter expressions. Making topological relations explicit in a subject-predicate-object structure could improve query performance, as they are less expensive than spatial queries. However, there is a trade-off with the required storage space. Furthermore, the chances of incorrect or broken links increase as both features and subsequently their topological relations can change over time. 

The presented conceptual system architecture automatically retrieves metadata from a \ac{sos}. However, it could also have been designed on top of a \ac{csw}: the metadata from a \ac{sos} could be harvested by the \ac{csw} and this \ac{csw} could be connected to a semantic knowledge base. The advantage of this method is that the \ac{csw} standard with \acs{sor} and \acs{sir} extensions could be reused. The method would then perform the same linked data conversions, but without the \ac{sos} harvesting. This approach has not been selected in this thesis because it also has a number of disadvantages. First of all, there would be more data duplication: Data would be stored at the source, duplicated by the \ac{csw} and afterwards again on the semantic web. Secondly, it places a greater burden on the organisation maintaining the \ac{sos}. The organisation would have to create a \ac{csw} next to their \ac{sos}, or make sure it is connected to a third party's \ac{csw}.  

\section{Recommendations}
\label{recommend}
Based on the conclusions of this thesis five recommendations have been formulated: 

\begin{itemize}
	\item \textit{The \ac{sos} standard should require a more descriptive capabilities document.}
	
	Currently a number of requests have to be made to find out which deployed sensors can be accessed via a \ac{sos}. For making sense of \ac{sos} metadata and to identify deployed sensors, it is crucial that the relations between sensor locations, procedures and observed properties are described. Therefore, a capabilities document should not only list \acp{foi}, procedures and observed properties, but it should also state their interrelationship.    
	
	
	\item \textit{The \ac{swe} standards specifications should contain stricter rules regarding the structure of \acp{uri}.}
	
	Observed properties and procedures need to be semantically defined by the organisation maintaining a \ac{sos}, for the presented processes to be completed automated. Semantic descriptions are not mandatory using the current \ac{swe} standards specifications. Therefore, a more strict \ac{swe} specification should be created. It should require the use of semantic \acp{uri}, which resolve to \ac{rdf} documents. 
	
	
	\item \textit{The \ac{om} Observation schema should be able to include aggregated sensor data with the same observed property, created by different sensing procedures.}
	
	The output data of the proof of concept implementation is not in accordance with the \ac{om} \ac{xml} schema, because it outputs spatially aggregated observation data. This data can be retrieved from different sources, each using different procedures. This does not fit in the current \ac{om} Observation schema. Therefore, the value of a procedure element should be allowed to include multiple procedures with the same observed property.   

	
	\item \textit{Linked data ontologies should be extended with classes for geo web services, supported requests and \ac{sos} content.}
	
	The om-lite and sam-lite ontologies cover a large part of the sensor metadata. However, these linked data ontologies should be extended to also define the \ac{sos} service, it's supported requests and additional metadata, such as observation offerings. These could not yet be semantically described in the proof of concept.
	
	
	\item \textit{The performance of endpoints should be improved, to better cope with vector geometries in \ac{sparql} queries}
	
	Since GeoSPARQL and stSPARQL require the verbose \ac{wkt} or \acs{gml} encodings for spatial filters, the queries can be rejected for exceeding the maximum amount of characters. The performance of endpoints should be improved, to allow more complex vector geometries to be used in \ac{sparql} queries. 
	
\end{itemize}




