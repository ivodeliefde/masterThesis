%!TEX root = thesis.tex

% ~2 pages 
\chapter{Discussion}
\label{chap:disc}

A method has been presented in this thesis and tested in a proof of concept implementation. In the process of designing this method and testing a number of design decisions had to be made. This chapter provides a brief reflection on each of these topics of discussion.

\section*{Metadata duplication}
The method presented in this thesis takes metadata from a \ac{sos}, converts it to linked data and publishes it on the semantic web. In essence the data is being stored twice in two different locations with this method. Having duplicated data  may not be desirable, because the duplicate can become out of sync with the corresponding data original at the source. Also more available storage space is required for the same amount of data. 

However, extra functionality is achieved in return, primarily with respect to data discovery. This why the \ac{csw} also creates duplicate data of metadata from \ac{ogc} geo web services. The \ac{sir} extension for the \ac{csw} has a fixed time interval that can be set, for performing regular updates. The current method presented in this thesis has not yet implemented an update mechanism. A solution for this could be to integrate the method into common \ac{sos} software packages to allow the \ac{sos} to update its own metadata on the semantic web. In this scenario an update will only take place when metadata is updated at the source.

The semantic sensor middleware solutions provide a method for outputing \ac{rdf} documents directly from a \ac{sos}. This does not require any data duplication, but essentially adds an extra output format to the \ac{sos} based on a linked data model. The trade of between duplication and functionality is also visible here: these approaches do not offer more functionality for discovering sensor metadata than the \ac{sos} already had.       

\section*{Metadata quality}
The method of Chapter \ref{chap:design} is aiming to automate as many steps in the chain of discovering and processing sensor data as possible. This includes a harvesting process which should be executed to automatically creating linked data from \ac{sos} metadata. The philosophy behind this is that an automated process is more convenient for users, because it saves them time and effort, and that it does not make any mistakes when performing the same task multiple times in a row. However, the metadata is not machine understandable if there is no meaning added at all to instances like observed properties inside the \ac{sos}. For example, a procedure represented by a five digit integer cannot be semantically interpreted. In this case, manual work has to be done to make it machine understandable. Only after this manual process it can be converted to linked data and published on the semantic web. 

Nevertheless, even if a \ac{sos} has provided the minimum of semantics for the \ac{wps} to understand its content, an automated process is still error prone. The quality of the metadata in a \ac{sos} influences the quality of the metadata at the \ac{sparql} endpoint. Since the method is designed to work with any \ac{sos} as input there lies a responsibility on the side of the organisations maintaining these services to provide sufficient, understandable and correct metadata.   

 

\section*{Explicit topological relations}
Spatial features have topological relations with other spatial features. These relations can be made explicit on the semantic web using ontologies such as GeoSPARQL. However, in this thesis they have not been made explicit and are calculated on-the-fly with spatial queries using stSPARQL filter expressions. Making topological relations explicit in a subject-predicate-object structure could improve query speed, as they are likely less expensive than spatial queries. However, this is a trade-off with the required storage space. Furthermore, the chances of incorrect or broken links increase as both features and subsequently their topological relations can change over time. 

\section*{The use of a catalog service}
The methods presented in Chapter \ref{chap:design} automatically retrieves metadata from a \ac{sos}. However, it could also have been designed on top of a \ac{csw}: the metadata from a \ac{sos} would be harvested by the \ac{csw} and this \ac{csw} would be connected to a semantic knowledge base. The advantage of this method would be that the \ac{csw} standard with \ac{sor} and \ac{sir} extensions could be reused as well. The method would then perform the same linked data conversions, but without the \ac{sos} harvesting. This approach has not been selected in this thesis because it also has a number of disadvantages. First of all, there would be even more data duplication than the current method. Data would be stored at the source, duplicated by the \ac{csw} and then duplicated again on the semantic web. Second of all, it places a greater burden on the organisation maintaining the \ac{sos}, as it would have to create a \ac{csw} next to it or make sure it is connected to a third parties \ac{csw}.  