%!TEX root = thesis.tex

% ~1 page
\chapter{Future Work}
\label{chap:futureResearch}

The method presented in this thesis has been created with a limited amount of time. Therefore, there are still a number of areas in which the method could be improved, and questions that should still be answered. This chapter will briefly go over each of these future research topics. 

\section*{Including more web services}
The current implementation has been created using the \ac{sos}. However, there are a number of other sensor web services based on \ac{swe} standards that could be included. First of all, there are the services that provide other kinds of functionalities besides data retrieval such as the \acf{sas}, \acf{sps} and \acf{wns}. The inclusion of these services would broaden the range of tasks a \ac{wps} could automatically perform. 

On the other hand, other services for observation data retrieval could be added as well, such as the tinySOS and the SensorThings \ac{api}. These are services for retrieving observation data from a \ac{iot} perspective, with a focus on low energy consumption by the `thing' that senses and more compact queries and responses than the verbose \ac{xml} documents that the \ac{sos} uses. Nevertheless, all of these services are based on the \ac{om} data model and therefore the principles applied to the \ac{sos} can also be applied to the other sensor data services.  

Next to sensor related web services the method could also be extended to include other \ac{ogc} geo web standards, such as the \acf{wfs} and \acf{wcs}. Similar to the \ac{sos} it's content could be made available in a machine understandable manner to allow them to be discovered and used in automated web processes.    

\section*{Improving performance}
The main focus of developing the current method was to define its functionality. The proof of concept implementation showed that the performance of discovering sensor metadata was good. However, the process that automatically queries the \aclp{sos} could still be made faster. Currently for every sensor a request is send. This approach could be improved by requesting all sensor data from the same \ac{sos} using fewer or perhaps even using only a single request. The available offerings of a \ac{sos} could play a role in this improvement.    

\section*{Extending linked data ontologies}
The ontologies that have been used in this thesis do not yet cover all concepts of sensor metadata. There are for example no semantic definitions of what \ac{ogc} geo web services are, such as \ac{sos}. They are now defined as merely a \ac{url}, but this does not at all describe the service, its allowed requests and responses. For this and other missing classes as described in Chapter \ref{chap:results} and \ref{chap:conclusion} linked data ontologies should be extended to properly define all sensor metadata.   

\section*{Extending the methods of aggregation}
\label{par:FRaggregation}

Basic aggregation methods have currently be implemented. However, as \cite{SW:Ganesan} points out spatial or temporal aggregation can give a distorted result if no weight are added to compensate for differences in density. This extension could therefore create more reliable outcomes produced by the \ac{wps}. 

Another useful way to extend the aggregation methods of this thesis, is to describe each of these methods semantically. \cite{SSW:Stasch4} proposes the use of semantic definitions of spatial aggregation. This could make the purpose of each these methods machine understandable, as well as the effect that it has on the input data. This could prevent an automated process of making a logical mistake that leads to meaningless output. 
