% !TeX root = ../../thesis.tex

% ~10 pages 

\chapter{Results}
\label{chap:results}

\section{Implementation differences between Sensor Observation Services}
Even though both \aclp{sos} use the \ac{swe} standards they are not exactly the same. They have different approaches for making identifiers and have slightly different content in their capabilities and describe sensor documents. This has caused the implemented to be more complex than expected. Especially providing a describe sensor document with as much information as possible (containing related features of interest, observed property and offerings) is important for making sense of the metadata. This should also be done using the appropriate \ac{sensorml} tags, instead of the more general `swe:keywords' for example. 

When retrieving sensor data from the \ac{sos} using \texttt{GetObservation} requests one implementation provides this data as an array of comma separated values, using the \ac{swe} Array Observation class. The other implementation provides the same kind of data embedded in \ac{xml} tags using the \ac{om} Measurement class.  

One of the \ac{sos} implementations allowed multiple observed properties to be related to an observation procedure. This was an unexpected finding, because a procedure is defined as a \enquote{method, algorithm or instrument, or system of these which may be used in making an observation} \cite[p. 4]{SW:ISO}. When a procedure has different observed properties it is rather a combination of procedures. It is unlikely that the same procedure can observe one property at a certain time and a completely other property at another time. The original implementation of the \ac{wps} had to be adjusted to deal with this.  

\section{Semantics in Sensor Observation Services}
\subsection{mapping of observable properties}
The metadata in \aclp{sos} does not necessarily contain semantics. This is an issue when automatically retrieving metadata from different services. In the implementation this has caused a problem with identifying which observable properties in a \ac{sos} are the same as observable properties found in another \ac{sos}.

\subsection{Automatically creating an URI scheme}
If an \ac{uri} is automatically created for data from an unknown source there is an amount of uncertainty to what the \ac{uri} will look like. Either a random identifier can be assigned to it, or the identifier that is already provided by the data source. This identifier most likely contains some reference to the nature of the real world thing the \ac{uri} represents and is therefore preferable to a a completely random identifier. Adding the given identifier to the \ac{uri} can create very long and strange \acp{uri} because different data sources have a different way of creating \acp{uri} since the \ac{om} schema allows simply `any \ac{uri}'.   

\section{Spatial queries with SPARQL}
\label{par:spQueries}
For retrieving data about a vector feature three methods for spatial querying have been implemented. First of all, spatial queries in which the server side receives the complete geometry of a feature and checks which point geometries it contains. Second of all, spatial queries in which the server side receives the bounding box of the complete geometry and check which points are within it. Third of all, spatial queries in which the server side receives the \ac{eea} reference grid cell that overlap the geometry and checks which points each cell contains. With the first method the client does not have to perform any spatial queries anymore. The second and third method will also return points that are not inside the geometry of the vector feature. These points could be filtered out by the client. 

The Strabon and Parliament endpoint have been tested since they both handle GeoSPARQL queries. Strabon has been used in the final design, because the Parliament endpoint rejected certain longer queries (see \ref{par:spQueries}).

Pubby software in combination with Apache Tomcat allows for a user interface that is easier to navigate through for humans. The links stored in \ac{rdf} triples are represented as hyperlinks which can be used to navigate between pages about different concepts. 

\subsection{Vector queries}
GeoSPARQL allows spatial queries in which a geometry can be inserted in the query. The implementation uses this functionality to test for spatial relations between geometries. For example between the point geometry of a sensor and the polygon geometry of an administrative unit. However, when geometries become more complicated their \ac{wkt} definition becomes more verbose. This leads to the query being rejected based on its number of characters by the endpoint. This indicates that vector queries with complex geometries are not very efficient to include in a \ac{sparql} query.     

\subsection{Raster queries}
All raster cells are retrieved that overlap with the vector geometry. For these raster cell sensors are retrieved and later the excess ones are filtered out. 

\subsection{Bounding box queries}
Creating a bounding box around a vector feature and make \texttt{GetObservation} to the \ac{sos}.  

\subsection{Latitude and longitude order}
During the implementation a problem has come up regarding the order in which latitude and longitude are being presented in the \ac{sos}. The \ac{sos} of the \ac{rivm} provides point geometries in WGS84 as longitude, latitude and height. However, the Strabon endpoint expects the order to be latitude, longitude and height. This results false outcomes of spatial queries.    

\section{Output data}
Data is outputted in \ac{xml} or \ac{json} according to the \ac{om} schema. However, this schema cannot be used when different processes have been used by the data that has been aggregated.

\section{Comparing the Sensor Instance Registry with a semantic knowledge base}
Comparing \cite{SW:OGC3} and \cite{SW:OGC4} with the outcomes of Chapter \ref{chap:impl}

\section{Comparing the semantic sensor middleware with a semantic knowledge base}
Comparing the outcomes of Chapter \ref{chap:impl} with methods for retrieving \ac{rdf} from a \ac{sos}
