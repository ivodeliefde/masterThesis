%!TEX root = thesis.tex

\chapter{Data}
\label{chap:data}

\begin{sloppypar}
	The data can be divided into static geographic data and sensor data. For static data the topographic data sets from the \ac{cbs} and the Dutch cadaster will be used. The dataset of Dutch provinces (provincies, Figure \ref{fig:provinces}) and municipalities (gemeenten, Figure \ref{fig:municipalities}) has been downloaded from \url{https://www.pdok.nl/nl/producten/pdok-downloads/basis-registratie-kadaster/bestuurlijke-grenzen-actueel}. It is difficult to obtain data of administrative boundaries of Belgium (even from the \ac{inspire} data portal). Therefore, all data for Belgium was retrieved from \url{http://www.gadm.org/}. The geographic data contains the name of the administrative units and their (polygon) geometry.
	
	Data on landcover will be used to complement the data of administrative units. A section of the 2012 dataset from the \ac{corine} programme will be used (Figure \ref{fig:CORINE}). This dataset contains polygons (Figure \ref{fig:CORINEZOOM}) with a unique identifier, a code that defines the type of landcover and the size of the polygon's surface. It was downloaded from \url{http://land.copernicus.eu/pan-european/corine-land-cover/clc-2012}. 
	
\end{sloppypar}

\begin{figure}
	\centering
	\includegraphics[width=0.5\linewidth]{figs/Municipalities.png}
	\caption{Dataset of municipalities in the Netherlands and Belgium in 2015 (from Dutch cadaster and GADM.org)}
	\label{fig:municipalities}
\end{figure}

\begin{figure}
	\centering
	\includegraphics[width=0.5\linewidth]{figs/Provinces.png}
	\caption{Dataset of provinces in the Netherlands and Belgium in 2015 (from Dutch cadaster and GADM.org)}
	\label{fig:provinces}
\end{figure}

\begin{figure}
	\centering
	\includegraphics[width=1\linewidth]{figs/CORINE_NL_BE_color.PNG}
	\caption{Dataset of landcover in the Netherlands and Belgium in 2012 (from Copernicus  The European Earth Observation Programme)}
	\label{fig:CORINE}
\end{figure}

\begin{figure}
	\centering
	\includegraphics[width=1\linewidth]{figs/CORINE_NL_BE_color_zoom.PDF}
	\caption{Landcover of the province of South Holland (subsection of the dataset from Figure \ref{fig:CORINE})}
	\label{fig:CORINEZOOM}
\end{figure}

Air quality sensor data will be used from the \ac{rivm} (\url{http://inspire.rivm.nl/sos/}) and from the \ac{ircel} (\url{http://sos.irceline.be/}). Both of these organisations have a \ac{sos} where data can be retrieved according to the \ac{swe} standards. The one of the \ac{rivm} has been online since the 21\textsuperscript{st} of August, 2015. \ac{ircel} already made the \ac{sos} available on the first of January, 2011. Figure \ref{fig:RIVMSensor} and Figure \ref{fig:IRCELINESensor} show the sensor networks of both organisations. They provide different kinds of sensor data, such as particulate matter ($PM_{10}$), nitrogen dioxide ($NO^{2}$) and ozone ($O^{3}$). Figure \ref{fig:Sensor} shows one of the sensor locations in the city center of Amsterdam. 


\begin{figure}
	\centering
	\includegraphics[width=0.6\linewidth]{figs/RIVMSensors.png}
	\caption{Webmap by the \ac{rivm} showing their air quality sensor network (\url{http://www.lml.rivm.nl/meetnet})}
	\label{fig:RIVMSensor}
\end{figure}

\begin{figure}
	\centering
	\includegraphics[width=0.6\linewidth]{figs/IRCELINESensors.png}
	\caption{Webmap by \ac{ircel} showing their air quality sensor network (\url{http://www.irceline.be/en/air-quality/measurements/monitoring-stations/})}
	\label{fig:IRCELINESensor}
\end{figure}

\begin{figure}
	\centering
	\includegraphics[width=1\linewidth]{figs/SensorAdam.png}
	\caption{Google Streetview image of \ac{rivm} sensor location in Amsterdam in 2015}
	\label{fig:Sensor}
\end{figure}
