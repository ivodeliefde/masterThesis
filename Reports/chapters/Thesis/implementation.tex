%!TEX root = thesis.tex

\chapter{Implementation}
\label{chap:impl}

Implementation of the methods are described in this chapter.

\section{Preparing linked data}
First linked data is prepared that will be used to retrieve and process sensor data on the semantic web (Figure \ref*{fig:Static}).

\begin{figure}
	%\centering
	\includegraphics[width=0.7\linewidth]{UML/staticdata2.PNG}
	\caption{Static Data}
	\label{fig:Static}
\end{figure}

\subsection{Creating linked data of administrative units}
Data of countries, provinces and municipalities

Using RDFlib triples are added for every administrative unit: 
\begin{itemize}
	\item administrative unit $>$ type $>$ dbpedia country/province/municipality/neighbourhood
	\item administrative unit $>$ foaf:Name $>$ Name
	\item administrative unit $>$ hasGeometry $>$ geometryObject
	\item geometryObject $>$ type $>$ geometry
	\item geometryObject $>$ dataType $>$ WKTLiteral
	\item geometryObject $>$ literal $>$ Polygon(....)
\end{itemize}

\subsection{Creating linked data of land cover}
\ac{corine} 2012 land cover dataset. 

Using RDFlib triples are added for every landcover feature: 
\begin{itemize}
	\item landcover feature $>$ type $>$ dbpedia landcover type definition
	\item landcover feature$>$ foaf:Name $>$ ID
	\item landcover feature$>$ hasGeometry $>$ geometryObject
	\item geometryObject $>$ type $>$ geometry
	\item geometryObject $>$ dataType $>$ WKTLiteral
	\item geometryObject $>$ literal $>$ Polygon(....)
\end{itemize}

\subsection{Creating linked raster data}
The \ac{eea} reference grid with resolutions of 10km\textsuperscript{2} and 100km\textsuperscript{2}.

\subsection{Publishing Resource Description Framework data}
Setting up the Apache Jena Fuseki (Figure \ref{fig:Fuseki}), Apache Tomcat and Pubby software (Figure \ref{fig:Pubby}\textbf{}).

\begin{figure}
	%\centering
	\includegraphics[width=\linewidth]{figs/fuseki.PNG}
	\caption{Apache Jena Fuseki endpoint}
	\label{fig:Fuseki}
\end{figure}

\begin{figure}
	%\centering
	\includegraphics[width=\linewidth]{figs/pubby.PNG}
	\caption{Pubby interface for \ac{rdf}}
	\label{fig:Pubby}
\end{figure}


\section{Retrieving metadata from the Sensor Observation Service}
The metadata is automatically retrieved from the \ac{sos}.

problems: 
\begin{itemize}
	\item Observation are the central concept in a \ac{sos}. This makes it hard to retrieve the locations of the sensors from the capabilities document. A workaround is to retrieve small amounts of observations for each feature of interest.
	\item Although standardised, the \ac{sos} contains information that can be implemented in different ways. This makes it hard to automatically retrieve data. An example is the feature of interest in the \texttt{GetObservation} document. 
\end{itemize}

\section{Modelling with the om-lite and sam-lite ontologies}
Using the ontologies to create a model that suits the purpose of this thesis.

\section{Establishing inward links from DBPedia}
Sending triples to DBPedia the project.

\section{Setting up the Web Processing Services}
Creating two \ac{wps} using PyWPS.

\section{Prototype implementation}
Crawling the semantic web to find specific sensor data.