% !TeX root = ../thesis.tex
\lstset{
	basicstyle=\ttfamily,
	columns=fullflexible,
	showstringspaces=false,
	commentstyle=\color{gray}\upshape
}

\lstdefinelanguage{XML}
{
	morestring=[b]",
	morestring=[s]{>}{<},
	morecomment=[s]{<?}{?>},
	stringstyle=\color{black},
	identifierstyle=\color{darkblue},
	keywordstyle=\color{cyan},
	morekeywords={xml,xsi,xmlns,type,xlink}% list your attributes here
}


\chapter{Web Processing Service Response Documents}
\label{app:WPS}

This appendix shows a number of examples of response documents by a \acf{wps}. The following sections show a \texttt{GetCapabilities} response (Section \ref{app:wpsCapabilities}), a \texttt{DescribeProcess} response (Section \ref{app:wpsDescribe}) and an \texttt{Execute} response (Section \ref{app:wpsExecute}). More information about the \ac{wps}, its requests and responses can be found in Section \ref{par:wps}.

\section{Example Capabilities Document}
\label{app:wpsCapabilities}
\begin{lstlisting}[language=xml]
<wps:Capabilities xmlns:xlink="http://www.w3.org/1999/xlink" xmlns:wps="http://www.opengis.net/wps/1.0.0" xmlns:ows="http://www.opengis.net/ows/1.1" xmlns:xsi="http://www.w3.org/2001/XMLSchema-instance" service="WPS" version="1.0.0" xml:lang="en-CA" xsi:schemaLocation="http://www.opengis.net/wps/1.0.0 http://schemas.opengis.net/wps/1.0.0/wpsGetCapabilities_response.xsd" updateSequence="1">
	<ows:ServiceIdentification>
		<ows:Title>PyWPS Server</ows:Title>
		<ows:Abstract>
			See http://pywps.wald.intevation.org and http://www.opengeospatial.org/standards/wps
		</ows:Abstract>
		<ows:Keywords>
			<ows:Keyword>WPS</ows:Keyword>
			<ows:Keyword>SWE</ows:Keyword>
			<ows:Keyword>SOS</ows:Keyword>
		</ows:Keywords>
		<ows:ServiceType>WPS</ows:ServiceType>
		<ows:ServiceTypeVersion>1.0.0</ows:ServiceTypeVersion>
		<ows:Fees>None</ows:Fees>
		<ows:AccessConstraints>none</ows:AccessConstraints>
	</ows:ServiceIdentification>
	<ows:ServiceProvider>
		<ows:ProviderName>Delft University of Technology</ows:ProviderName>
		<ows:ProviderSite xlink:href="http://masterthesistudelft.herokuapp.com/"/>
		<ows:ServiceContact>
		<ows:IndividualName>Ivo de Liefde</ows:IndividualName>
		<ows:PositionName>MSc. student Geomatics for the Built Environment</ows:PositionName>
		<ows:ContactInfo>
			<ows:Address>
			<ows:DeliveryPoint>Julianalaan 134</ows:DeliveryPoint>
			<ows:City>Delft</ows:City>
			<ows:PostalCode>2628 BL</ows:PostalCode>
			<ows:Country>the Netherlands</ows:Country>
			<ows:ElectronicMailAddress>i.deliefde@student.tudelft.nl</ows:ElectronicMailAddress>
			</ows:Address>
			<ows:OnlineResource xlink:href="http://masterthesistudelft.herokuapp.com/"/>
			<ows:HoursOfService>0:00-24:00</ows:HoursOfService>
			<ows:ContactInstructions>none</ows:ContactInstructions>
		</ows:ContactInfo>
		<ows:Role>
			Created WPS
		</ows:Role>
		</ows:ServiceContact>
	</ows:ServiceProvider>
	<ows:OperationsMetadata>
		<ows:Operation name="GetCapabilities">
			<ows:DCP>
				<ows:HTTP>
					<ows:Get xlink:href="http://localhost/cgi-bin/wps?"/>
					<ows:Post xlink:href="http://localhost/cgi-bin/wps"/>
				</ows:HTTP>
			</ows:DCP>
		</ows:Operation>
		<ows:Operation name="DescribeProcess">
			<ows:DCP>
				<ows:HTTP>
					<ows:Get xlink:href="http://localhost/cgi-bin/wps?"/>
					<ows:Post xlink:href="http://localhost/cgi-bin/wps"/>
				</ows:HTTP>
			</ows:DCP>
		</ows:Operation>
		<ows:Operation name="Execute">
			<ows:DCP>
				<ows:HTTP>
					<ows:Get xlink:href="http://localhost/cgi-bin/wps?"/>
					<ows:Post xlink:href="http://localhost/cgi-bin/wps"/>
				</ows:HTTP>
			</ows:DCP>
		</ows:Operation>
	</ows:OperationsMetadata>
	<wps:ProcessOfferings>
		<wps:Process wps:processVersion="1.0">
			<ows:Identifier>LinkedDataFromSOS</ows:Identifier>
			<ows:Title>Creates Linked Data of SOS metadata</ows:Title>
			<ows:Abstract>
				This process takes an HTTP address of a Sensor Observation Service (SOS) as input and converts the metadata to linked data.
			</ows:Abstract>
		</wps:Process>
		<wps:Process wps:processVersion="1.0">
			<ows:Identifier>GetSensors</ows:Identifier>
			<ows:Title>
				Automatically retrieves sensors from heterogenous sources using the semantic web
			</ows:Title>
			<ows:Abstract>
				This process takes a sensor data request with parameters for spatial features of interest, observed property, temporal range and granularity, and finds all relevant sensor data sources on the semantic web.
			</ows:Abstract>
		</wps:Process>
		<wps:Process wps:processVersion="1.0">
			<ows:Identifier>GetSensorData</ows:Identifier>
			<ows:Title>
				Automatically retrieves, integrates and aggregates heterogenous sensor data using the semantic web
			</ows:Title>
			<ows:Abstract>
			This process takes sensors found by the WPS 'GetSensors' and automatically integrates and aggregates the data from different sources on the web.
			</ows:Abstract>
		</wps:Process>
	</wps:ProcessOfferings>
	<wps:Languages>
		<wps:Default>
			<ows:Language>en-CA</ows:Language>
		</wps:Default>
		<wps:Supported>
			<ows:Language>en-CA</ows:Language>
		</wps:Supported>
	</wps:Languages>
	<wps:WSDL xlink:href="http://localhost/cgi-bin/wps?WSDL"/>
</wps:Capabilities>
\end{lstlisting}

\clearpage

\section{Example Describe Process Document}
\label{app:wpsDescribe}
\begin{lstlisting}[language=xml]
	<wps:ProcessDescriptions xmlns:wps="http://www.opengis.net/wps/1.0.0" xmlns:ows="http://www.opengis.net/ows/1.1" xmlns:xlink="http://www.w3.org/1999/xlink" xmlns:xsi="http://www.w3.org/2001/XMLSchema-instance" xsi:schemaLocation="http://www.opengis.net/wps/1.0.0 http://schemas.opengis.net/wps/1.0.0/wpsDescribeProcess_response.xsd" service="WPS" version="1.0.0" xml:lang="en-CA">
	<ProcessDescription wps:processVersion="1.0" storeSupported="true" statusSupported="false">
		<ows:Identifier>LinkedDataFromSOS</ows:Identifier>
		<ows:Title>Creates Linked Data of SOS metadata</ows:Title>
		<ows:Abstract>
			This process takes an HTTP address of a Sensor Observation Service (SOS) as input and converts the metadata to linked data.
		</ows:Abstract>
		<DataInputs>
			<Input minOccurs="0" maxOccurs="1">
				<ows:Identifier>observed_properties</ows:Identifier>
				<ows:Title>
					Input link to turtle file with mappings of observed property identifiers to DBPedia URIs
				</ows:Title>
				<LiteralData>
					<ows:DataType ows:reference="http://www.w3.org/TR/xmlschema-2/#string">string</ows:DataType>
					<ows:AnyValue/>
					<DefaultValue>http://inspire.rivm.nl/sos/eaq/service?</DefaultValue>
				</LiteralData>
			</Input>
			<Input minOccurs="0" maxOccurs="1">
				<ows:Identifier>input_url</ows:Identifier>
				<ows:Title>
					Input a string containing an HTTP address of a Sensor Observation Service (SOS). For example: 'http://someaddress.com/sos?'
				</ows:Title>
				<LiteralData>
					<ows:DataType ows:reference="http://www.w3.org/TR/xmlschema-2/#string">string</ows:DataType>
					<ows:AnyValue/>
					<DefaultValue>http://inspire.rivm.nl/sos/eaq/service?</DefaultValue>
				</LiteralData>
			</Input>
		</DataInputs>
	</ProcessDescription>
</wps:ProcessDescriptions>
\end{lstlisting}

\clearpage

\section{Example Execute Document}
\label{app:wpsExecute}
\begin{lstlisting}[language=xml]
<wps:ExecuteResponse xmlns:wps="http://www.opengis.net/wps/1.0.0" xmlns:ows="http://www.opengis.net/ows/1.1" xmlns:xlink="http://www.w3.org/1999/xlink" xmlns:xsi="http://www.w3.org/2001/XMLSchema-instance" xsi:schemaLocation="http://www.opengis.net/wps/1.0.0 http://schemas.opengis.net/wps/1.0.0/wpsExecute_response.xsd" service="WPS" version="1.0.0" xml:lang="en-CA" serviceInstance="http://localhost/cgi-bin/wps?service=WPS&request=GetCapabilities&version=1.0.0" statusLocation="http://localhost/wps/wpsoutputs/pywps-146072575066.xml">
	<wps:Process wps:processVersion="1.0">
		<ows:Identifier>GetSensors</ows:Identifier>
		<ows:Title>
			Automatically retrieves sensors from heterogenous sources using the semantic web
		</ows:Title>
		<ows:Abstract>
			This process takes a sensor data request with parameters for spatial features of interest, observed property, temporal range and granularity, and finds all relevant sensor data sources on the semantic web.
		</ows:Abstract>
	</wps:Process>
	<wps:Status creationTime="2016-04-15T15:09:52Z">
		<wps:ProcessSucceeded>PyWPS Process GetSensors successfully calculated</wps:ProcessSucceeded>
	</wps:Status>
	<wps:ProcessOutputs>
		<wps:Output>
			<ows:Identifier>output</ows:Identifier>
			<ows:Title>Output sensor data</ows:Title>
			<wps:Data>
				<wps:ComplexData mimeType="text/JSON">
					OUTPUT JSON DATA
				</wps:ComplexData>
			</wps:Data>
		</wps:Output>
	</wps:ProcessOutputs>
</wps:ExecuteResponse>
\end{lstlisting}
