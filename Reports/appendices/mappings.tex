\chapter{Mappings}
\label{mappings}

This appendix provides an overview of the mappings, as described in chapter \ref{chap:design} and \ref{chap:impl}. Table \ref{map:SOS} shows how instances have been described from a provenance perspective. Table \ref{map:getFOI} present mappings of GetFeatureOfInterest and GetCapabilities response documents to classes of the om-lite and sam-lite ontologies. More information on how these mappings were used to create linked data from sensor metadata can be found in Subsection \ref{}. How these mappings have been used in \ac{sparql} queries can be found in Subsection \ref{par:discoverSensors} and \ref{par:discoverSensors}.

\begin{table}[!htbp]
	\centering
	\caption{Mapping of SOS instances using PROV-O}
	\label{map:SOS}
	\begin{tabular}{l|l}
		Instance		      		  	& Mapped to                           \\ \hline
		SOS  						  	& http://www.w3.org/ns/prov\#Agent     \\
		Sensor                         	& http://www.w3.org/ns/prov\#Agent     \\
		Procedure                       & http://www.w3.org/ns/prov\#Activity  \\
		FOI                           	& http://www.w3.org/ns/prov\#Entity                                                                 
	\end{tabular}
\end{table}

\begin{sidewaystable}[!htbp]
	\centering
	\caption{GetFeatureOfInterest mapping using om-lite and sam-lite}
	\label{map:getFOI}
	\resizebox{0.8\columnwidth}{!}{%
	\begin{tabular}{l|l|l}
		GetFeatureOfInterest content    & Represents       & Mapped to                                                         \\ \hline
		sams:SF\_SpatialSamplingFeature & Sampling feature & http://def.seegrid.csiro.au/ontology/om/sam-lite\#SamplingFeature \\
		sams:shape                      & Sampling point   & http://def.seegrid.csiro.au/ontology/om/sam-lite\#SamplingPoint   \\
		ns:pos                          & Point geometry   & \multirow{2}{*}{http://strdf.di.uoa.gr/ontology\#hasGeometry}     \\
		ns:pos{[}srsName{]}             & CRS              &                                                                  
	\end{tabular}
}
\end{sidewaystable}

\begin{sidewaystable}[!htbp]
	\centering
	\caption{Mappings of capabilities document using om-lite and sam-lite}
	\label{map:capabilities}
	\resizebox{\columnwidth}{!}{%
		\begin{tabular}{l|l|l}
			Capabilities content                                           & Represents                                   & Mapped to                                                            \\ \hline
			ows:Title                                                      & Name of SOS                                  & http://xmlns.com/foaf/0.1/name								         \\
			ows:ProviderName                                               & Name of organisation maintaining SOS         & http://xmlns.com/foaf/0.1/name                                       \\
			ows:Parameter{[}@name='featureOfInterest'{]}/ows:AllowedValues & Identifiers of all FOIs                      & http://def.seegrid.csiro.au/ontology/om/sam-lite\#SamplingPoint      \\
			ows:Parameter{[}@name='procedure'{]}/ows:AllowedValues         & Identifiers of all procedures                & http://def.seegrid.csiro.au/ontology/om/om-lite\#Process             \\
			sos:ObservationOffering                                        & Offerings                                    & http://def.seegrid.csiro.au/ontology/om/sam-lite\#SamplingCollection \\
			swes:observableProperty                                        & Observable property of offering              & http://def.seegrid.csiro.au/ontology/om/om-lite\#observedProperty    \\
			swes:procedure                                                 & Procedure of offering                        & http://def.seegrid.csiro.au/ontology/om/om-lite\#Process             \\
			ows:Parameter{[}@name='responseFormat'{]}/ows:AllowedValues    & Identifiers of response formats              & http://purl.org/dc/terms/hasFormat                                   \\
			ows:Parameter{[}@name="procedureDescriptionFormat"{]}          & Identifiers of procedure description formats & http://purl.org/dc/terms/hasFormat                                  
		\end{tabular}
	}
\end{sidewaystable}




